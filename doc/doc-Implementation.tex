
\section{Implementation}
\label{sec:implementation}

For information on implementing file systems under Linux see the file \filename{vfs.txt} in the
\filename{Documentation/filesystems} directory of the kernel source tree. That document
describes the VFS and some of the things that need to be done to support it\footnote{Keep in
  mind that the documentation in \filename{Documentation}, like all Linux kernel documentation
  you might find, is not necessarily up to date. Always cross-check what the documentation tells
  you against the source code.}. You can also look at some of the file systems that have already
been implemented. For example the directory \filename{fs/ext2} contains the implementation of
the ext2 file system.

It is generally much easier to implement a file system for read-only access than it is to handle
the fully general case. This is because file systems that can only be read don't need to support
the methods that modify file system structures, and they don't need to worry about consistency,
locking, dirty cache buffers, and so forth. Thus, the initial implementation of \GenericFS\
provides just read-only operation.

There are several steps involved in implementing a new file system for Linux. However, not all
steps need to be implemented before the system becomes useful. The VFS will provide reasonable
defaults for some operations, and you can either just return errors from or provide no
implementation for the others until you support them. In this way a preliminary implementation
of \GenericFS\ can be loaded into the kernel and used, perhaps with reduced functionality, even
before all aspects of it are finished.

There are several operations structures that need to be initialized with pointers to actual
functions. I will call these functions ``methods.'' The \code{super_block} structure contains
methods for looking up and disposing of inodes. The \code{inode_operations} structure contains
methods for dealing with individual inodes. The \code{file_operations} structure contains
methods for dealing with open files. These methods and the supporting functions they require,
constitute the bulk of the driver code.

\subsection{The Disk Tool}
\label{sec:implementation-disktool}

To facilitate testing the \GenericFS\ distribution includes a user mode program that can modify
a disk partition directly to build various file system data structures. This tool allows you to
``manually'' construct a \GenericFS\ file system for use by the kernel module even if the kernel
module is not in a state where it can write \GenericFS\ structures reliably. The tool also helps
you debug the kernel module. After making a change to the module and then running a test case,
you can unmount the \GenericFS\ partition and use the disk tool to inspect the \GenericFS\
structures that the module created or modified.

The disk tool takes the name of a disk partition on the command line and presents a menu of
options using a curses based interface. The options are mostly self-explanatory. Note that the
disk tool combines the functionality of a file system creation tool, a file system checking
tool, and a file system dumping tool, all in an interactive package. While this is nice for
experimenting with \GenericFS\ stand-alone versions of these tools should also be written at
some point for better scriptability.

Note that the disk tool accesses the partition directly by opening the block device file.
\emph{Never run the disk tool on a mounted partition!} This approach bypasses the usual file
system handling code and lets the disk tool communicate directly with the disk cache. In this
way the tool can inspect and modify file system data structures arbitrarily.

The disk tool uses an \code{ioctl} call on the open block device in order to find out the size
of the partition. The code in Figure \ref{fig:get-size} illustrates this.

\begin{figure*}[tp]
  \centering
  \begin{wbigbox}
\begin{lstlisting}{}
/* Determine the partition size in blocks. */
if (ioctl(fd, BLKGETSIZE, &size) >= 0) {
  block_count = size / (BLOCKSIZE / 512);
  printf("%ld blocks; %ld bytes (4K blocks)\n",
    block_count, block_count * BLOCKSIZE);
}
else {
  printf("Can't figure out partition size!\n");
  return 1;
}
\end{lstlisting}
  \end{wbigbox}
  \caption{Getting Partition Size}
  \label{fig:get-size}
\end{figure*}

Here \code{block_count} is a long integer representing the number of blocks on the partition.
\code{BLKGETSIZE} is a special number defined in \code{<sys/ioctl.h>}. The \code{ioctl} call
returns the size of the partition in 512 byte sectors. \code{BLOCKSIZE} is 4 KiB in our case.

Note that if the partition is not a multiple of 4 KiB the last few bytes of the partition will
be unused. The partition size is always a multiple of 512 bytes since it contains an integer
number of sectors.

You can also use a program such as \filename{od} directly on the partition's device file to view
the data in the partition. However, \filename{od} does not understand \GenericFS, of course, and
thus won't display the data in an easily understood manner. However, it can be useful for
debugging the disk tool itself. See the \filename{od} manual page for more information.

\subsubsection*{Exercises}

\begin{enumerate}

\item The disk tool should never be used on a partition that has been mounted. Why not?

\item Enhance the current disk tool. For example

  \begin{enumerate}
    \item Enhance or extend one of the existing options.
    \item Implement one of the currently unimplemented options.
    \item Add a new option.
    \item Add the ability to edit \GenericFS\ data structures as well as
      display them.
  \end{enumerate}
  
\item Implement a disk tool program for some other file system such as ext2.

\item Implement stand-alone \GenericFS\ tools for making, checking, and dumping a \GenericFS\
  file system. Review the ext2 tools for doing this for ideas on what they might do and how they
  might work.

\end{enumerate}

\subsection{Mount a \GenericFS\ Partition}
\label{sec:implementation-mount}

The main file of the \GenericFS\ driver module is \filename{super.c}. It is in this file that
the super block methods are located. This includes the necessary code to support mounting a
\GenericFS\ partition. The central structure in this file is \code{gfs_type} of type struct
\code{file_system_type}. The definition of struct \code{file_system_type} is in
\filename{linux/fs.h}. The name used in that structure is the type name used with the
\filename{mount} command's -t option. The flag \code{FS_REQUIRES_DEV} forces the file system
onto a device. The function \code{gfs_read_super} reads the super block out of a given partition
and is used during the mount operation.

Note that the \code{kill_sb} member of the file system structure is invoked by the kernel when a
partition is unmounted. It points at a kernel helper function that takes care of some
administrative tasks and then calls the \code{put_super} and \code{write_super} super block
methods.

The module initializer, \code{init_module}, registers \GenericFS\ with the kernel. Once the
driver is inserted, the file \filename{/proc/filesystems} should list ``genericfs'' as one of
the supported file system types due to this registration.

The super block reader function just uses a kernel helper routine to do the dirty work of
initializing a \code{super_block} structure. The helper routine then invokes
\code{gfs_fill_super} to read the \GenericFS\ super block and prepare the \GenericFS\ specific
in memory data structures. The basic idea is to use \code{bread} to read the block containing
the super block, verify that the super block is in the right format for \GenericFS, fill in some
fields of the VFS's \code{super_block} structure, and allocate an inode for the root directory
of the partition. Getting the root directory will cause the \code{read_inode} super block method
to be invoked.

When reading data from the disk into variables in memory the helper functions \code{le16_to_cpu}
and \code{le32_to_cpu}. These functions convert little endian data to the native CPU format. For
little endian CPUs they are macros that do nothing. However, if the module is ever compiled on a
big endian CPU these functions will do an important translation. \GenericFS\ mandates little
endian data on disk, but we still want big endian processors to be able to handle it.

One issue that comes up when reading inodes is that there is more information in the disk inode
than can be put into the members of the VFS inode. This is because the VFS is not interested in
how the data is laid out on the disk. Each file system manages that in its own way. The VFS
can't provide the necessary members to hold layout specific information since the methods used
by different file systems are likely to be radically different.

To deal with this, the inode structures in memory are actually larger than the VFS inode. In
particular each \GenericFS\ inode is represented using a \code{gfs_inode_info} structure. This
structure contains \GenericFS\ specific members and a VFS inode member as well. Yet the VFS
deals only with the VFS inode enclosed in each \code{gfs_inode_info}.

Since there will likely be many inodes in the inode cache it is important to allocate the
\code{gfs_inode_info} structures efficiently. Thus, the driver creates a kernel SLAB cache for
this purpose. A SLAB cache is a cache optimized for allocation of many objects that are all the
same size. There would be a different SLAB cache for different sized objects; at any point in
time the kernel has several SLAB caches active (see \filename{/proc/slabinfo}). Figure
\ref{fig:init-slab} shows how a SLAB cache is created for the inode structures.

\begin{figure*}[tp]
  \centering
  \begin{wbigbox}
\begin{lstlisting}{}
kmem_cache_t *gfs_icache;

int init_module(void)
{
  gfs_icache = kmem_cache_create(
      "gfs_inode_icache",
      sizeof(struct gfs_inode_info),
      0,
      SLAB_RECLAIM_ACCOUNT | SLAB_MEM_SPREAD,
      init_once);

  if (gfs_icache == 0) {
    printk(KERN_ERR "GenericFS: Can't allocate inode cache.\n");
    return -ENOMEM;
  }
  /* The rest of init_module. */
}

void cleanup_module(void)
{
  /* The rest of cleanup_module. */
  if (!gfs_icache) kmem_cache_destroy(gfs_icache);
}
\end{lstlisting}    
  \end{wbigbox}
  \caption{Initializing and Removing a SLAB Cache}
  \label{fig:init-slab}
\end{figure*}

Figure \ref{fig:init-slab} shows the pointer to \code{kmem_cache_t} as a global variable. This
is necessary so that the other functions in the module are able to access it. Note especially
that the SLAB cache should not be stored in a mounted partition's super block since the same
SLAB cache can be used for all mounted \GenericFS\ partitions.

A pointer to an inode initialization function \code{init_once} is also passed to
\code{kmem_cache_create}. The SLAB cache calls this function for each object it allocates.
In the current implementation of \GenericFS\ this function simply calls a kernel helper function
\code{inode_init_once} to take care of the VFS inode member. In the future it might also be
necessary to initialize the \GenericFS\ specific parts of the inode structure.

\subsection{Read the Root Directory}
\label{sec:implementation-readroot}

In this and following sections I will use the phrase ``root directory'' to mean the top level
directory on a \GenericFS\ partition.

After one can successfully mount and unmount a \GenericFS\ partition the next step is to
implement enough code so that the command \command{ls -l} returns appropriate information about
the files stored in the root directory.

When a user mode process tries to scan a directory it calls the POSIX functions \code{opendir},
\code{readdir}, and \code{closedir}. In Linux these functions are not system calls but rather
library functions. In particular, \code{opendir} actually invokes the \code{open} system call
using the \code{O_DIRECTORY} flag to open the directory being scanned. The \code{readdir}
function then calls \code{getdents} to get directory entries, several at a time. It turns out
that in the implementation of \code{open} the existence of the \code{lookup} method in the
directory's inode operations is checked. If the method is \code{NULL} then \code{open} assumes
that the file being opened is not a directory and it returns the ``Not a directory'' error code.
This activity can be witnessed if you write a short demonstration program that scans a directory
and then use the \command{strace} command on that program.

At this stage we don't need to provide an actual \code{lookup} method, but to satisfy
\code{open} you will need to at least provide a placeholder method so that you can install a
non-NULL pointer in the directory inode operations structure.

To satisfy the \code{getdents} system call we need to implement the \code{readdir} method in the
file operations structure.

\subsection{Read Files}
\label{sec-implementation-readfiles}

\subsection{Read Metadata}
\label{sec-implementation-readmeta}

\subsection{Read Subdirectories}
\label{sec-implementation-readsubdirs}

\subsection{Write Files}
\label{sec-implementation-writefiles}

\subsection{Write Metadata}
\label{sec-implementation-writemeta}

\subsection{Write Subdirectories}
\label{sec-implementation-writesubdirs}

\subsection{Special Files}
\label{sec-implementation-special}
