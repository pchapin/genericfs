%%%%%%%%%%%%%%%%%%%%%%%%%%%%%%%%%%%%%%%%%%%%%%%%%%%%%%%%%%%%%%%%%%%%%%%%%%%%
% FILE   : doc-GenericFS-Structure.tex
% SUBJECT: The on-disk layout of GenericFS
% AUTHOR : (C) Copyright 2013 by Peter C. Chapin
%
% TODO:
%
% + See comments in the text.
%
% Send comments or bug reports to:
%
%    Peter C. Chapin
%    Computer Information Systems
%    Vermont Technical College
%    Williston, VT 05495
%    PChapin@vtc.vsc.edu
%%%%%%%%%%%%%%%%%%%%%%%%%%%%%%%%%%%%%%%%%%%%%%%%%%%%%%%%%%%%%%%%%%%%%%%%%%%%

\section{\GenericFS\ Structure}
\label{sec:structure}

In this section I will describe the layout of \GenericFS\ on the disk. Keep in mind that the
layout of file system data structures in memory is not necessarily the same as the on-disk
layout. In memory you will need to abide by the requirements of the Virtual File System (VFS)
and store information into VFS structures as appropriate. In cases where the VFS is not specific
about the in-memory layout (for example for the free maps) you should consider designs that
offer good efficiency even if the on-disk data needs to be significantly reformatted when it is
brought into memory.

In the discussion that follows when I display structures I will assume, unless otherwise noted,
that there is no padding space between the elements. The driver uses appropriate types and
packing declarations to ensure that the compiler generates the right layout no matter what
processor is being targeted.

In addition \GenericFS\ stores all multibyte quantities on the disk in little endian form. The
driver uses byte swapping helper functions in the kernel to insure that the endianness of the
data is correct when loading or storing disk data into kernel data structures.

\subsection{General Layout}
\label{sec:structure-general}

\GenericFS\ block size is always 4 KBytes. Furthermore there is one inode for every block. As a
consequence of this design a \GenericFS\ partition will never normally run out of inodes (do you
see that?). These constraints may be lifted in the future and thus there are places where the
size of a block and the number of inodes on a partition are treated as adjustable parameters
as a step in ``future-proofing'' the system. However under the current design the size of all
file system metadata is entirely depending on the size of the partition.

The zeroth block on the partition is the super block. It declares the partition as \GenericFS\
and provides information on the size and layout of the metadata. It serves no other purpose and
most of the superblock is unusued. The layout of the super block follows the structure in Figure
\ref{fig:super-layout}.

\begin{figure}[htbp]
  \centering
  \begin{bigbox}
\begin{lstlisting}{}
struct gfs_superblock {
  uint32_t magic_number;
  uint32_t block_size;
  uint32_t total_blocks;
  uint32_t inodefreemap_blocks;
  uint32_t blockfreemap_blocks;
  uint32_t inodetable_blocks;
};
\end{lstlisting}
  \end{bigbox}
  \caption{Superblock layout}
  \label{fig:super-layout}
\end{figure}

The magic number is 0xDEADBEEF. This identifies the partition as a \GenericFS\ partition. The
block size is currently always 4096. The other fields count the number of blocks in each of the
file system areas described below.

Following the super block is the inode free map. It is a bit map where the zeroth bit of the
zeroth byte represents the zeroth inode. If an inode's bit is a one that implies that the inode
is being used. The inode free map is an integer number of blocks. The last block is unlikely to
be fully used since the number of inodes on the disk is unlikely to be an exact multiple of the
number of bits in a block.

Following the inode free map is the block free map. It has exactly the same size and format as
the inode free map. The fact that both of these free maps have the same size is a manifestation
of the ``one inode per block'' constraint.

Following the block free map is the inode table itself. It contains space for all of the inodes.
Each inode is 64 bytes. Since 64 divides 4 KBytes, no inode will overlap a block
boundary\footnote{Thankfully. Managing such ``split inodes'' would be very awkward}. The inode
table is an integer number of blocks in size. Part of the last block is unlikely to be fully
used.

\subsection{Inodes}
\label{sec:structure-inode}

Inodes have the format shown in Figure \ref{fig:inode-layout}. The fields are mostly self
explanatory. Notice that block numbers are unsigned 32 bits. Notice also that the numbers for
the first four blocks of the file are stored in the inode itself. Only two indirection pointers
are used.

\begin{figure}[htbp]
  \centering
  \begin{bigbox}
\begin{lstlisting}{}
struct gfs_inode {
  uint32_t nlinks;
  uint32_t owner_id;
  uint32_t group_id;
  uint32_t mode;
  uint32_t file_size;
  uint32_t atime;
  uint32_t mtime;
  uint32_t ctime;
  uint32_t blocks[4];
  uint32_t first_indirect;
  uint32_t second_indirect;
  uint32_t unused[2];
};
\end{lstlisting}
  \end{bigbox}
  \caption{Inode layout}
  \label{fig:inode-layout}
\end{figure}

This structure allows for a maximum file size of $(4 + 4096/4 + (4096/4)^2) \times 4096$ bytes.
This works out to 4,299,177,984 bytes. This is actually slightly greater than $2^{32} - 1$
(which is 4,294,967,295 bytes). Since the file size is represented with a 32 bit quantity, it is
the $2^{32} - 1$ that actually limits the maximum file size. This is one reason why the system
does not use a 3rd indirection pointer (another reason is that it adds unnecessary complications
to what is intended to be a simple file system). In theory the largest disk this system could
handle would be 4K*4G bytes ($2^{32}$ blocks, each 4~Kbytes in size). This is 16 TiB. In
practice current\footnote{2002-09} Linux block device drivers use signed 32 bit integers to
number disk sectors. Since disk sectors are usually 512 bytes, this implies a maximum partition
size of $2^{31} \times 512 = 1$ Tbyte.

\subsubsection*{Exercises}

\begin{enumerate}

\item Suppose a future version of \GenericFS\ supported block sizes of 2 Kbytes, 4 Kbytes, or 8
  Kbytes. What would be the size of the largest file in each of those cases?

\item Could the \GenericFS\ inode format be extended to hold 64~bit file sizes? Would there be
  any point in doing this? Discuss.

\item It is desirable to write the driver and any \GenericFS\ tools to use only the information
  in the superblock to locate file system metadata. Why not instead take advantage of known
  properties of \GenericFS\ (such as the fact that the number of inodes equals the number of
  blocks) to simplify the programming?

\end{enumerate}

\subsection{Directories}
\label{sec:structure-directories}

Immediately following the inode table is the first block of the root directory. The root
directory is described by inode \#0 and it is set up when the file system is initialized. It is
important to keep in mind that directories are treated as special kinds of files. The disk space
they occupy is described by an inode the same as for any other file.

The directory structure is shown in Figure~\ref{fig:directory-layout}. Prefixing each directory
entry is a 32 bit quantity that contains the offset of the next valid directory entry relative
to the beginning of the directory file. An offset of zero implies that there are no more
directory entries. No normal directory entry should have a next offset field of zero because the
directory entry at offset zero is the special ``.'' entry and that entry is never removed. This
also implies that the first entry in the directory is always at offset zero.

\begin{figure}[tbhp]
  \center
  \scalebox{0.40}{\includegraphics*{Figures/fig-GenericFS-Directory-Layout.pdf}}
  \caption{Layout of a GenericFS Directory}
  \label{fig:directory-layout}
\end{figure}

As a consequence of this design the directory entries form a singly linked list inside the
directory file. There is no particular restriction on the spacing of directory entries in the
directory file---there could be gaps between entries. However, to keep the directory format
manageable the list never flows backwards in the directory file. The offset of an entry is
always beyond the offset of its previous entry. Also, no directory entry ever crosses a block
boundary in the directory file. Thus it is never necessary to read two blocks to see a single
directory entry.

Immediately after the next offset pointer, each directory entry consists of an unsigned 32 bit
inode number followed by an eight bit quantity that specifies the length of the file's name,
where zero specifies a length of 256 characters. Following the length information are the
characters of the file name itself. The name need not be null terminated. In fact, as far as
\GenericFS\ is concerned names could contain embedded null characters or slash characters.
However, \GenericFS\ does not currently support Unicode names explicitly. Names are assumed to
be composed of ASCII characters and the use of arbitrary binary data in file names results in
undefined behavior.

\subsubsection*{Exercises}

\begin{enumerate}

\item The specification above requires that the list of directory entries in a directory only
  flow forward. Does that really provide any advantage? If so what? What would be involved in
  implmenting a more general approach where the list is allowed to flow in either direction?

\item As files are created and removed holes will develop in the directory file. Will the holes
  ever be used again? Is there any value in periodically compacting directories to remove those
  holes?

\item Looking up an entry in a linked list is an $O(n)$ operation. For large directories with
  many files that could be time consuming. Two alternative schemes are using B-trees or hash
  tables. Consider how these alternatives might be implemented. Which would be easier? What
  advantages would they provide? Can the \GenericFS\ software be modularized so that directory
  handling could be easily replaced later (say for experimentation purposes)?

\end{enumerate}
