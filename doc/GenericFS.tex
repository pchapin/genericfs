%%%%%%%%%%%%%%%%%%%%%%%%%%%%%%%%%%%%%%%%%%%%%%%%%%%%%%%%%%%%%%%%%%%%%%%%%%%%
% FILE   : GenericFS.tex
% SUBJECT: Description of the Generic File System for Linux
% AUTHOR : (C) Copyright 2024 by Peter C. Chapin
%
% This document describes how to build, debug, and test file system modules in Linux using the
% GenericFS file system as a complete example. Readers are invited to experiment with GenericFS
% and several suggestions for such experiments are given in exercises.
%
% TODO:
%
% + See comments in the text.
%
% Send comments or bug reports to:
%
%    Peter Chapin
%    Vermont State University
%    Williston, VT 05495
%    peter.chapin@vermontstate.edu
%%%%%%%%%%%%%%%%%%%%%%%%%%%%%%%%%%%%%%%%%%%%%%%%%%%%%%%%%%%%%%%%%%%%%%%%%%%%

%+++++++++++++++++++++++++++++++++
% Preamble and global declarations
%+++++++++++++++++++++++++++++++++
\documentclass[twocolumn]{article}
\usepackage{listings}
\usepackage[pdftex]{graphics}
\usepackage{hyperref}
\usepackage{inconsolata}
\usepackage{color}

\setlength{\parindent}{0em}
\setlength{\parskip}{1.75ex plus0.5ex minus0.5ex}

% I really should put this in a package file! Note how I set up some parameters before opening
% the listing. This used to be necessary when I was using verbatim environments. Is it still
% necessary with the listings package? Are those other settings just being overridden?
%
\newsavebox{\savebigbox}
\newenvironment{bigbox}{\begin{lrbox}{\savebigbox}
  \begin{minipage}{0.95\columnwidth}%
    \small\setlength{\baselineskip}{0.9\baselineskip}}
{\end{minipage}\end{lrbox}\fbox{\usebox{\savebigbox}}}

% This version should be used for full width boxes.
\newsavebox{\savewbigbox}
\newenvironment{wbigbox}{\begin{lrbox}{\savewbigbox}
  \begin{minipage}{0.9\textwidth}%
    \small\setlength{\baselineskip}{0.9\baselineskip}}
{\end{minipage}\end{lrbox}\fbox{\usebox{\savewbigbox}}}

% Various macros.
\newcommand{\newterm}[1]{\emph{#1}}
\newcommand{\code}[1]{\lstinline@#1@}
\newcommand{\filename}[1]{\texttt{#1}}
\newcommand{\command}[1]{\texttt{#1}}
\newcommand{\GenericFS}{\texttt{GenericFS}}
\newcommand{\todo}[1]{\textit{TODO: #1}}

% The following are settings for the listings package.
\lstdefinestyle{customconsolas}{
    basicstyle=\ttfamily, % Use \ttfamily to switch to monospaced font
    keywordstyle=\color{blue}\bfseries,
    commentstyle=\color{green},
    stringstyle=\color{red},
    showstringspaces=false,
    breaklines=true,
    columns=fullflexible,
    basicstyle=\fontsize{10}{11}\selectfont\ttfamily
}
\lstset{language=C, style=customconsolas}


%++++++++++++++++++++
% The document itself
%++++++++++++++++++++
\begin{document}

%-----------------------
% Title page information
%-----------------------
\title{GenericFS Documentation}
\author{\copyright\ Copyright 2024 by Peter Chapin}
\date{July 13, 2024}
\maketitle

\tableofcontents

\section*{Legal}
\label{sec:legal}

\textit{Permission is granted to copy, distribute and/or modify this document under the terms of
the GNU Free Documentation License, Version 1.1 or any later version published by the Free
Software Foundation; with no Invariant Sections, with no Front-Cover Texts, and with no
Back-Cover Texts. A copy of the license is included in the file \texttt{GFDL.txt} distributed
with the \LaTeX\ source of this document.}

\section{Introduction}
\label{sec:intro}

\textbf{generic}. \textit{A product, such as a drug or detergent, that is sold without a brand
name or trademark. --The American Heritage Dictionary}

This document contains a detailed description of the \GenericFS\ Linux file system. \GenericFS\
is a simple file system with no special features or extra, unnecessary complications. It is
intended to serve primarily as an educational tool. By studying or enhancing \GenericFS\ you can
learn about Linux kernel programming, Linux file systems specifically, and file systems in
general.

To understand this description of \GenericFS\ you will need to be familiar with C programming.
You should also know something about how to use Linux, and be comfortable working with the Linux
command line shell. If you want to experiment with \GenericFS\ you will need a machine on which
you can be root so that you can partition disks, load kernel modules, and do other
administrative tasks. You should avoid using a machine that you depend on. \GenericFS\ is
experimental. It may crash your kernel. The accompanying documents \textit{Compiling Linux} and
\textit{DevBox and HackBox} contain information that will help you set up a virtual environment
with a suitable Linux kernel for experimentation.


\section{Preliminaries}
\label{sec:preliminaries}

It is not my intention to discuss kernel programming in detail here. There are other sources of
information about that subject that you could consult, including the \textit{Linux Kernel
Development} guide that is included in the \GenericFS\ documentation set. However, I do want to
talk about the mechanics of compiling and testing \GenericFS. If you are only interested in
installing \GenericFS\ and do not intend to modify or enhance it, you should still read the
subsection below on compiling \GenericFS. Otherwise, you can skip the rest of this section.

\subsection{Compiling \GenericFS}
\label{sec:compiling}

\GenericFS\ comes in two parts: a kernel module that contains the file system driver and a
``disk tool'' that can be used to initialize (format), read, edit, and check \GenericFS\
partitions. These two components are compiled individually.

The module build process is highly integrated with the kernel build system. You need to use a
properly constructed Makefile if you are to build a module consistently. The Makefile provided
with the \GenericFS\ module assumes that you will be compiling the driver for use with the same
kernel that is executing at the time you do the compilation. If this is not the case you will
need to modify the Makefile to reflect your intended paths. Furthermore, you must first
configure the kernel sources against which you are building your module. More information about
configuring the kernel and compiling modules can be found in the \textit{Linux Kernel
Development} companion document.

To compile \GenericFS\ change into the \filename{driver} directory and issue the command
\command{make}. The driver will be left in the file \filename{genericfs.ko}.

To compile the disk tool change into the \filename{disktool} directory and issue the command
\command{make}. The disk tool will be left in the executable file \filename{disktool}.

\subsection{Setting Up a \GenericFS\ Partition}
\label{sec:partition}

To test \GenericFS\ you will need a free disk partition. The size is not critical, although it
would be nice if it was large enough to exercise the file system reasonably. You can put
\GenericFS\ on a physical disk partition, or you can prepare a file system image stored in an
ordinary file. The second method is more flexible and probably the best choice for
experimentation.

\subsubsection{Physical Partition}

Use the \command{fdisk} command to create a partition on a physical disk or on a virtual disk
created by virtualization software such as VirtualBox. You may have to reboot your machine after
creating the new partition, so the kernel will see the new partition table. Look for the device
files for the new disk and its partitions in the \filename{/dev} directory.

\todo{Describe the process of using \command{fdisk} in more detail.}

\subsubsection{File System Image}

If you do not have any unpartitioned disk space on your system, you can still create a partition
for \GenericFS\ by using the loop back driver. This allows you to treat an ordinary file like a
block device. Once configured you can make a \GenericFS\ file system inside this file just as
you would on any other partition. Furthermore, once you load the \GenericFS\ driver you'll be
able to mount the file system inside this file as well. Another advantage to this approach is
that you can copy the file containing the \GenericFS\ partition to another computer for
inspection or experimentation. This is easier than moving physical (or even virtual) disk drives
around.

To create an image file, first create a file of some suitable size using the \command{dd}
command. For example, to create a 192 MiB file use a command such as:
\begin{verbatim}
$ dd if=/dev/zero of=disk.img count=393216
\end{verbatim}

Here the input file is the special file \filename{/dev/zero} which returns an endless stream of
zero bytes. The output file is \filename{disk.img} in this example, but it could have any name.
The count is given in units of 512 bytes, so a count of 393216 corresponds to an output file of
exactly 192 MiB.

The significance of this size is that it is large enough to cause the free maps to span multiple
blocks (specifically, 1.5 blocks each) without creating a disk image that is unduly large. This
is important for testing purposes. The details of \GenericFS's on-disk layout are discussed in
Section~\ref{sec:structure}.

\subsubsection{Formatting}

To create a \GenericFS\ file system on your new partition or image file you should use an
appropriate user mode tool. An interactive tool of this sort, called \command{disktool}, is
included in the \GenericFS\ package and is described in
Section~\ref{sec:implementation-disktool}. It plays the role of both \command{mkfs} and
\command{fsck}. It also gives you a way of creating and viewing \GenericFS\ data structures on
your partition and thus can be helpful for testing and debugging the \GenericFS\ driver.

Start \command{disktool} providing the name of the partition on the command line. This name is
either the device file representing the physical partition or the name of a previously created
image file, such as \filename{disk.img} as described above. The disk tool treats both kinds of
``partitions'' the same way. Use the Initialize menu option to format the partition.

The disk tool also sets every byte on the partition to 0x55. This takes time and is unnecessary
in general. However, it makes debugging file system data structures a little easier by erasing
any previous and potentially confusing information. Note that 0x55 is used instead of 0x00 so
the ``unused'' bytes stand out more in the hex dumps provided by \command{disktool}.

\subsubsection{Mounting}

Before \GenericFS\ can be used it is necessary to load the \GenericFS\ driver. In the
\filename{driver} directory, as root, use the command:
\begin{verbatim}
$ sudo insmod genericfs.ko
\end{verbatim}

This command inserts the \GenericFS\ module. You can check to see if the driver loaded correctly
by inspecting the contents of \filename{/proc/filesystems}. The ``genericfs'' file system should
be included in the list (probably at the bottom).

Once the image file is formatted and once the \GenericFS\ driver has been loaded into the
kernel, you can mount the image file, as root, using \emph{one} of the commands below
\begin{verbatim}
$ sudo mount /dev/partition gfs
$ sudo mount -o loop disk.img gfs
\end{verbatim}

The first command is appropriate if you created a physical partition for \GenericFS. The second
command is appropriate if you created an image file. It uses the loop back driver to mount the
file system inside the image file. Use only one of the two commands shown, depending on your
situation.

The name \filename{gfs} is the mount point. It should be a pre-existing empty directory. After
executing the mount command, you should be able to change into the \filename{gfs} directory and
use the files there normally.

\subsubsection{Unmounting}

Before you can manipulate a \GenericFS\ partition directly with the disk tool you must first
unmount it. This can be done, as root, with the command:
\begin{verbatim}
$ sudo umount gfs
\end{verbatim}

Here \filename{gfs} is the name of the mount point.

It is not necessary to unload the \GenericFS\ driver when you unmount a partition. You should be
able to unmount and then re-mount a partition multiple times without touching the driver.
However, if you make a change to the driver (and recompile it) you will need to unload and then
reload the driver before your change will be visible to your system. Be aware that all
\GenericFS\ partitions need to be unmounted before you can unload the driver. Unload the driver
with the command:
\begin{verbatim}
$ sudo rmmod genericfs
\end{verbatim}

Notice that you don't include the \filename{.ko} extension on the name when you unload the
driver. Technically you \command{insmod} a file but \command{rmmod} a driver. Thus,
\command{insmod} requires the name of a file, but \command{rmmod} only expects the driver's
name.

\subsubsection*{Exercises}

\begin{enumerate}

\item The \command{fdisk} program will assign partition type 83 (``Linux'') to any new partition
  by default. Is this really an appropriate type for an experimental file system?

\item What would be a reasonable size for an experimental \GenericFS\ partition? Justify your
  answer. You may need to consult information about the \GenericFS\ on-disk layout in
  Section~\ref{sec:structure}.

\item Set up a file system image file and try formatting the image file using a standard file
  system such as ext4. Try mounting it and copying some files to it.

\item Would there be any complications in putting a file for the loop back driver on a
  \GenericFS\ partition and then building a \GenericFS\ file system inside that file? Think
  about what happens inside the \GenericFS\ driver. Try it.

\end{enumerate}

\subsection{Debugging \GenericFS}
\label{sec:debugging}

Debugging kernel modules is difficult because they are not normal processes and thus can't be
controlled by a debugger. There are a number of techniques you can use, however.

\subsubsection{Basic Debugging Techniques}

The first technique is to simply include extra \code{printk} calls in your module. You can then
observe which of these calls are triggered by looking for the output in the system log file (or
in other places depending on the configuration of \command{syslog} and the priority level you
give the \code{printk}). This technique can be effective, but it won't help you if the functions
in your module are not getting called. Also, you should be sure to enclose your \code{printk}
statements with conditional compilation directives so that they can be turned on and off by
simply recompiling your module with appropriate options.

In fact, there is currently a facility in the \GenericFS\ source for producing and managing
debugging messages this way. The details are in \filename{global.h}, but the specific macros of
interest are shown in Figure~\ref{fig:debugging-macros}.

\begin{figure*}[htbp]
  \centering
  \begin{wbigbox}
\begin{lstlisting}{}
#define DEBUG_HEADER KERN_INFO "GenericFS DEBUG: %s: "

#if DEBUG_LEVEL == 0
  #define GENERIC_DEBUG(level, statement)
#else
  #define GENERIC_DEBUG(level, statement) \
    if (DEBUG_LEVEL >= level) { statement; }
#endif

#define ENTERED \
  GENERIC_DEBUG(1, printk(DEBUG_HEADER "Entered\n", __FUNCTION__))
\end{lstlisting}
  \end{wbigbox}
  \caption{\GenericFS\ Debugging Macros (in \filename{global.h})}
  \label{fig:debugging-macros}
\end{figure*}

Several ``debugging levels'' can be used with these macros. Level one is the default and is
intended to output messages when each significant \GenericFS\ function is entered. This is
easily accomplished by using the \texttt{ENTERED} macro at the top of each such function. A
debug level of zero removes the debugging messages entirely. Higher debug levels are intended to
trace progressively finer details in the execution of the module. This is implemented by using
the \texttt{GENERIC\-\_DEBUG} macro where appropriate. To change the debug level one must modify
its definition in \filename{global.h} and recompile the module. In a future version of the
\GenericFS\ driver, the debug level might be dynamically configurable.

With the debugging messages enabled you can gain insight on the operation of the \GenericFS\
driver by observing what happens when a particular user mode program executes. For example, to
help understand how \GenericFS\ services a particular system call, write a program that invokes
that system call and then see which functions (if any) in the \GenericFS\ driver get invoked.
The program \command{strace} can help you find precisely which system calls your test program
uses. You should use a tool like \command{strace} instead of just guessing what system calls are
invoked because many apparent ``system'' calls are actually library functions built on top of
the true system calls\footnote{POSIX does not specify which functions are system calls and which
  are library functions. POSIX only requires that all functions it specifies be supported.}.

\subsubsection{Kernel Debugger}

\todo{Talk about using the kernel debugger.}

%%%%%%%%%%%%%%%%%%%%%%%%%%%%%%%%%%%%%%%%%%%%%%%%%%%%%%%%%%%%%%%%%%%%%%%%%%%%
% FILE   : doc-GenericFS-Structure.tex
% SUBJECT: The on-disk layout of GenericFS
% AUTHOR : (C) Copyright 2013 by Peter C. Chapin
%
% TODO:
%
% + See comments in the text.
%
% Send comments or bug reports to:
%
%    Peter C. Chapin
%    Computer Information Systems
%    Vermont Technical College
%    Williston, VT 05495
%    PChapin@vtc.vsc.edu
%%%%%%%%%%%%%%%%%%%%%%%%%%%%%%%%%%%%%%%%%%%%%%%%%%%%%%%%%%%%%%%%%%%%%%%%%%%%

\section{\GenericFS\ Structure}
\label{sec:structure}

In this section I will describe the layout of \GenericFS\ on the disk. Keep in mind that the
layout of file system data structures in memory is not necessarily the same as the on-disk
layout. In memory you will need to abide by the requirements of the Virtual File System (VFS)
and store information into VFS structures as appropriate. In cases where the VFS is not specific
about the in-memory layout (for example for the free maps) you should consider designs that
offer good efficiency even if the on-disk data needs to be significantly reformatted when it is
brought into memory.

In the discussion that follows when I display structures I will assume, unless otherwise noted,
that there is no padding space between the elements. The driver uses appropriate types and
packing declarations to ensure that the compiler generates the right layout no matter what
processor is being targeted.

In addition \GenericFS\ stores all multibyte quantities on the disk in little endian form. The
driver uses byte swapping helper functions in the kernel to insure that the endianness of the
data is correct when loading or storing disk data into kernel data structures.

\subsection{General Layout}
\label{sec:structure-general}

\GenericFS\ block size is always 4 KBytes. Furthermore there is one inode for every block. As a
consequence of this design a \GenericFS\ partition will never normally run out of inodes (do you
see that?). These constraints may be lifted in the future and thus there are places where the
size of a block and the number of inodes on a partition are treated as adjustable parameters
as a step in ``future-proofing'' the system. However under the current design the size of all
file system metadata is entirely depending on the size of the partition.

The zeroth block on the partition is the super block. It declares the partition as \GenericFS\
and provides information on the size and layout of the metadata. It serves no other purpose and
most of the superblock is unusued. The layout of the super block follows the structure in Figure
\ref{fig:super-layout}.

\begin{figure}[htbp]
  \centering
  \begin{bigbox}
\begin{lstlisting}{}
struct gfs_superblock {
  uint32_t magic_number;
  uint32_t block_size;
  uint32_t total_blocks;
  uint32_t inodefreemap_blocks;
  uint32_t blockfreemap_blocks;
  uint32_t inodetable_blocks;
};
\end{lstlisting}
  \end{bigbox}
  \caption{Superblock layout}
  \label{fig:super-layout}
\end{figure}

The magic number is 0xDEADBEEF. This identifies the partition as a \GenericFS\ partition. The
block size is currently always 4096. The other fields count the number of blocks in each of the
file system areas described below.

Following the super block is the inode free map. It is a bit map where the zeroth bit of the
zeroth byte represents the zeroth inode. If an inode's bit is a one that implies that the inode
is being used. The inode free map is an integer number of blocks. The last block is unlikely to
be fully used since the number of inodes on the disk is unlikely to be an exact multiple of the
number of bits in a block.

Following the inode free map is the block free map. It has exactly the same size and format as
the inode free map. The fact that both of these free maps have the same size is a manifestation
of the ``one inode per block'' constraint.

Following the block free map is the inode table itself. It contains space for all of the inodes.
Each inode is 64 bytes. Since 64 divides 4 KBytes, no inode will overlap a block
boundary\footnote{Thankfully. Managing such ``split inodes'' would be very awkward}. The inode
table is an integer number of blocks in size. Part of the last block is unlikely to be fully
used.

\subsection{Inodes}
\label{sec:structure-inode}

Inodes have the format shown in Figure \ref{fig:inode-layout}. The fields are mostly self
explanatory. Notice that block numbers are unsigned 32 bits. Notice also that the numbers for
the first four blocks of the file are stored in the inode itself. Only two indirection pointers
are used.

\begin{figure}[htbp]
  \centering
  \begin{bigbox}
\begin{lstlisting}{}
struct gfs_inode {
  uint32_t nlinks;
  uint32_t owner_id;
  uint32_t group_id;
  uint32_t mode;
  uint32_t file_size;
  uint32_t atime;
  uint32_t mtime;
  uint32_t ctime;
  uint32_t blocks[4];
  uint32_t first_indirect;
  uint32_t second_indirect;
  uint32_t unused[2];
};
\end{lstlisting}
  \end{bigbox}
  \caption{Inode layout}
  \label{fig:inode-layout}
\end{figure}

This structure allows for a maximum file size of $(4 + 4096/4 + (4096/4)^2) \times 4096$ bytes.
This works out to 4,299,177,984 bytes. This is actually slightly greater than $2^{32} - 1$
(which is 4,294,967,295 bytes). Since the file size is represented with a 32 bit quantity, it is
the $2^{32} - 1$ that actually limits the maximum file size. This is one reason why the system
does not use a 3rd indirection pointer (another reason is that it adds unnecessary complications
to what is intended to be a simple file system). In theory the largest disk this system could
handle would be 4K*4G bytes ($2^{32}$ blocks, each 4~Kbytes in size). This is 16 TiB. In
practice current\footnote{2002-09} Linux block device drivers use signed 32 bit integers to
number disk sectors. Since disk sectors are usually 512 bytes, this implies a maximum partition
size of $2^{31} \times 512 = 1$ Tbyte.

\subsubsection*{Exercises}

\begin{enumerate}

\item Suppose a future version of \GenericFS\ supported block sizes of 2 Kbytes, 4 Kbytes, or 8
  Kbytes. What would be the size of the largest file in each of those cases?

\item Could the \GenericFS\ inode format be extended to hold 64~bit file sizes? Would there be
  any point in doing this? Discuss.

\item It is desirable to write the driver and any \GenericFS\ tools to use only the information
  in the superblock to locate file system metadata. Why not instead take advantage of known
  properties of \GenericFS\ (such as the fact that the number of inodes equals the number of
  blocks) to simplify the programming?

\end{enumerate}

\subsection{Directories}
\label{sec:structure-directories}

Immediately following the inode table is the first block of the root directory. The root
directory is described by inode \#0 and it is set up when the file system is initialized. It is
important to keep in mind that directories are treated as special kinds of files. The disk space
they occupy is described by an inode the same as for any other file.

The directory structure is shown in Figure~\ref{fig:directory-layout}. Prefixing each directory
entry is a 32 bit quantity that contains the offset of the next valid directory entry relative
to the beginning of the directory file. An offset of zero implies that there are no more
directory entries. No normal directory entry should have a next offset field of zero because the
directory entry at offset zero is the special ``.'' entry and that entry is never removed. This
also implies that the first entry in the directory is always at offset zero.

\begin{figure}[tbhp]
  \center
  \scalebox{0.40}{\includegraphics*{Figures/fig-GenericFS-Directory-Layout.pdf}}
  \caption{Layout of a GenericFS Directory}
  \label{fig:directory-layout}
\end{figure}

As a consequence of this design the directory entries form a singly linked list inside the
directory file. There is no particular restriction on the spacing of directory entries in the
directory file---there could be gaps between entries. However, to keep the directory format
manageable the list never flows backwards in the directory file. The offset of an entry is
always beyond the offset of its previous entry. Also, no directory entry ever crosses a block
boundary in the directory file. Thus it is never necessary to read two blocks to see a single
directory entry.

Immediately after the next offset pointer, each directory entry consists of an unsigned 32 bit
inode number followed by an eight bit quantity that specifies the length of the file's name,
where zero specifies a length of 256 characters. Following the length information are the
characters of the file name itself. The name need not be null terminated. In fact, as far as
\GenericFS\ is concerned names could contain embedded null characters or slash characters.
However, \GenericFS\ does not currently support Unicode names explicitly. Names are assumed to
be composed of ASCII characters and the use of arbitrary binary data in file names results in
undefined behavior.

\subsubsection*{Exercises}

\begin{enumerate}

\item The specification above requires that the list of directory entries in a directory only
  flow forward. Does that really provide any advantage? If so what? What would be involved in
  implmenting a more general approach where the list is allowed to flow in either direction?

\item As files are created and removed holes will develop in the directory file. Will the holes
  ever be used again? Is there any value in periodically compacting directories to remove those
  holes?

\item Looking up an entry in a linked list is an $O(n)$ operation. For large directories with
  many files that could be time consuming. Two alternative schemes are using B-trees or hash
  tables. Consider how these alternatives might be implemented. Which would be easier? What
  advantages would they provide? Can the \GenericFS\ software be modularized so that directory
  handling could be easily replaced later (say for experimentation purposes)?

\end{enumerate}


\section{Implementation}
\label{sec:implementation}

For information on implementing file systems under Linux see the file \filename{vfs.txt} in the
\filename{Documentation/filesystems} directory of the kernel source tree. That document
describes the VFS and some of the things that need to be done to support it\footnote{Keep in
  mind that the documentation in \filename{Documentation}, like all Linux kernel documentation
  you might find, is not necessarily up to date. Always cross-check what the documentation tells
  you against the source code.}. You can also look at some of the file systems that have already
been implemented. For example the directory \filename{fs/ext2} contains the implementation of
the ext2 file system.

It is generally much easier to implement a file system for read-only access than it is to handle
the fully general case. This is because file systems that can only be read don't need to support
the methods that modify file system structures, and they don't need to worry about consistency,
locking, dirty cache buffers, and so forth. Thus, the initial implementation of \GenericFS\
provides just read-only operation.

There are several steps involved in implementing a new file system for Linux. However, not all
steps need to be implemented before the system becomes useful. The VFS will provide reasonable
defaults for some operations, and you can either just return errors from or provide no
implementation for the others until you support them. In this way a preliminary implementation
of \GenericFS\ can be loaded into the kernel and used, perhaps with reduced functionality, even
before all aspects of it are finished.

There are several operations structures that need to be initialized with pointers to actual
functions. I will call these functions ``methods.'' The \code{super_block} structure contains
methods for looking up and disposing of inodes. The \code{inode_operations} structure contains
methods for dealing with individual inodes. The \code{file_operations} structure contains
methods for dealing with open files. These methods and the supporting functions they require,
constitute the bulk of the driver code.

\subsection{Mount a \GenericFS\ Partition}
\label{sec:implementation-mount}

The main file of the \GenericFS\ driver module is \filename{super.c}. It is in this file that
the super block methods are located. This includes the necessary code to support mounting a
\GenericFS\ partition. The central structure in this file is \code{gfs_type} of type struct
\code{file_system_type}. The definition of struct \code{file_system_type} is in
\filename{linux/fs.h}. The name used in that structure is the type name used with the
\filename{mount} command's -t option. The flag \code{FS_REQUIRES_DEV} forces the file system
onto a device. The function \code{gfs_read_super} reads the super block out of a given partition
and is used during the mount operation.

Note that the \code{kill_sb} member of the file system structure is invoked by the kernel when a
partition is unmounted. It points at a kernel helper function that takes care of some
administrative tasks and then calls the \code{put_super} and \code{write_super} super block
methods.

The module initializer, \code{init_module}, registers \GenericFS\ with the kernel. Once the
driver is inserted, the file \filename{/proc/filesystems} should list ``genericfs'' as one of
the supported file system types due to this registration.

The super block reader function just uses a kernel helper routine to do the dirty work of
initializing a \code{super_block} structure. The helper routine then invokes
\code{gfs_fill_super} to read the \GenericFS\ super block and prepare the \GenericFS\ specific
in memory data structures. The basic idea is to use \code{bread} to read the block containing
the super block, verify that the super block is in the right format for \GenericFS, fill in some
fields of the VFS's \code{super_block} structure, and allocate an inode for the root directory
of the partition. Getting the root directory will cause the \code{read_inode} super block method
to be invoked.

When reading data from the disk into variables in memory the helper functions \code{le16_to_cpu}
and \code{le32_to_cpu}. These functions convert little endian data to the native CPU format. For
little endian CPUs they are macros that do nothing. However, if the module is ever compiled on a
big endian CPU these functions will do an important translation. \GenericFS\ mandates little
endian data on disk, but we still want big endian processors to be able to handle it.

One issue that comes up when reading inodes is that there is more information in the disk inode
than can be put into the members of the VFS inode. This is because the VFS is not interested in
how the data is laid out on the disk. Each file system manages that in its own way. The VFS
can't provide the necessary members to hold layout specific information since the methods used
by different file systems are likely to be radically different.

To deal with this, the inode structures in memory are actually larger than the VFS inode. In
particular each \GenericFS\ inode is represented using a \code{gfs_inode_info} structure. This
structure contains \GenericFS\ specific members and a VFS inode member as well. Yet the VFS
deals only with the VFS inode enclosed in each \code{gfs_inode_info}.

Since there will likely be many inodes in the inode cache it is important to allocate the
\code{gfs_inode_info} structures efficiently. Thus, the driver creates a kernel SLAB cache for
this purpose. A SLAB cache is a cache optimized for allocation of many objects that are all the
same size. There would be a different SLAB cache for different sized objects; at any point in
time the kernel has several SLAB caches active (see \filename{/proc/slabinfo}). Figure
\ref{fig:init-slab} shows how a SLAB cache is created for the inode structures.

\begin{figure*}[tp]
  \centering
  \begin{wbigbox}
\begin{lstlisting}{}
kmem_cache_t *gfs_icache;

int init_module(void)
{
  gfs_icache = kmem_cache_create(
      "gfs_inode_icache",
      sizeof(struct gfs_inode_info),
      0,
      SLAB_RECLAIM_ACCOUNT | SLAB_MEM_SPREAD,
      init_once);

  if (gfs_icache == 0) {
    printk(KERN_ERR "GenericFS: Can't allocate inode cache.\n");
    return -ENOMEM;
  }
  /* The rest of init_module. */
}

void cleanup_module(void)
{
  /* The rest of cleanup_module. */
  if (!gfs_icache) kmem_cache_destroy(gfs_icache);
}
\end{lstlisting}    
  \end{wbigbox}
  \caption{Initializing and Removing a SLAB Cache}
  \label{fig:init-slab}
\end{figure*}

Figure \ref{fig:init-slab} shows the pointer to \code{kmem_cache_t} as a global variable. This
is necessary so that the other functions in the module are able to access it. Note especially
that the SLAB cache should not be stored in a mounted partition's super block since the same
SLAB cache can be used for all mounted \GenericFS\ partitions.

A pointer to an inode initialization function \code{init_once} is also passed to
\code{kmem_cache_create}. The SLAB cache calls this function for each object it allocates.
In the current implementation of \GenericFS\ this function simply calls a kernel helper function
\code{inode_init_once} to take care of the VFS inode member. In the future it might also be
necessary to initialize the \GenericFS\ specific parts of the inode structure.

\subsection{Read the Root Directory}
\label{sec:implementation-readroot}

In this and following sections I will use the phrase ``root directory'' to mean the top level
directory on a \GenericFS\ partition.

After one can successfully mount and unmount a \GenericFS\ partition the next step is to
implement enough code so that the command \command{ls -l} returns appropriate information about
the files stored in the root directory.

When a user mode process tries to scan a directory it calls the POSIX functions \code{opendir},
\code{readdir}, and \code{closedir}. In Linux these functions are not system calls but rather
library functions. In particular, \code{opendir} actually invokes the \code{open} system call
using the \code{O_DIRECTORY} flag to open the directory being scanned. The \code{readdir}
function then calls \code{getdents} to get directory entries, several at a time. It turns out
that in the implementation of \code{open} the existence of the \code{lookup} method in the
directory's inode operations is checked. If the method is \code{NULL} then \code{open} assumes
that the file being opened is not a directory and it returns the ``Not a directory'' error code.
This activity can be witnessed if you write a short demonstration program that scans a directory
and then use the \command{strace} command on that program.

At this stage we don't need to provide an actual \code{lookup} method, but to satisfy
\code{open} you will need to at least provide a placeholder method so that you can install a
non-NULL pointer in the directory inode operations structure.

To satisfy the \code{getdents} system call we need to implement the \code{readdir} method in the
file operations structure.

\subsection{Read Files}
\label{sec-implementation-readfiles}

\subsection{Read Metadata}
\label{sec-implementation-readmeta}

\subsection{Read Subdirectories}
\label{sec-implementation-readsubdirs}

\subsection{Write Files}
\label{sec-implementation-writefiles}

\subsection{Write Metadata}
\label{sec-implementation-writemeta}

\subsection{Write Subdirectories}
\label{sec-implementation-writesubdirs}

\subsection{Special Files}
\label{sec-implementation-special}


\bibliographystyle{plain}
\bibliography{references}

\end{document}
