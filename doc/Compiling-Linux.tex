%%%%%%%%%%%%%%%%%%%%%%%%%%%%%%%%%%%%%%%%%%%%%%%%%%%%%%%%%%%%%%%%%%%%%%%%%%%
% FILE    : Compiling-Linux.tex
% AUTHOR  : (C) Copyright 2024 by Peter Chapin
% SUBJECT : Compiling Linux for kernel development.
%
% This document describes how to set up a Linux kernel for kernel study and development. It
% covers how to compile the kernel and how to set up various kernel debugging and code browsing
% tools.
%
% TODO:
%
% + See comments in the text.
%
% Send comments or bug reports to:
%
%       Peter Chapin
%       Computer Information Systems
%       Vermont State University
%       141 Lawrence Pl.
%       Williston, VT. 05495
%       peter.chapin@vermontstate.edu
%%%%%%%%%%%%%%%%%%%%%%%%%%%%%%%%%%%%%%%%%%%%%%%%%%%%%%%%%%%%%%%%%%%%%%%%%%%

% ++++++++++++++++++++++++++++++++
% Preamble and global declarations
% ++++++++++++++++++++++++++++++++
\documentclass{article}

\usepackage{hyperref}
\usepackage{url}

% \pagestyle{headings}
\setlength{\parindent}{0em}
\setlength{\parskip}{1.75ex plus0.5ex minus0.5ex}

% ------------
% New Commands
% ------------

% Add commands in alphabetical order.
\newcommand{\command}[1]{\texttt{#1}}
\newcommand{\filename}[1]{\texttt{#1}}
\newcommand{\newterm}[1]{\textit{#1}}
\newcommand{\todo}[1]{\textit{TODO: #1}}

% ----------------
% New Environments
% ----------------

% An environment to display a sequence of commands.
\newenvironment{commands}
  {\begin{quote} \tt}
  {\end{quote}}


% +++++++++++++++++++
% The document itself
% +++++++++++++++++++
\begin{document}

% ----------------------
% Title page information
% ----------------------
\title{Compiling Linux}
\author{\copyright\ Copyright 2024 by Peter Chapin}
\date{Last Revised: June 7, 2024}
\maketitle

% -----------------
% Table of contents
% -----------------
\pagenumbering{roman}
\tableofcontents
\newpage
\pagenumbering{arabic}

%
% TODO: Switch to an appropriate Creative Commons license.
%
%\section*{Legal}
%\label{sec:legal}
%
%\textit{Permission is granted to copy, distribute and/or modify this document under the terms of
%  the GNU Free Documentation License, Version 1.1 or any later version published by the Free
%  Software Foundation; with no Invariant Sections, with no Front-Cover Texts, and with no
%  Back-Cover Texts. A copy of the license is included in the file \texttt{GFDL.txt} distributed
%  with the \LaTeX\ source of this document.}

\section{Introduction}

This document describes how to compile a Linux kernel for study and development. It contains
specific instructions for compiling an experimental kernel and for configuring various kernel
debugging and code browsing tools. This document was originally prepared to support the
operating systems class at Vermont State University. Some of the information it contains targets
that audience. However, much of the contents of this document would be useful to anyone
interested in studying the Linux kernel.

It is possible to run an experimental kernel on the same system as is being used for
development. However, this arrangement is not ideal since errors in the experimental kernel may
cause corruption of the development environment. Ideally, then, one should configure two
computers: a development system running a pre-built kernel and where all the programming tools
execute (compilers, editors, etc.), and a system for hacking that runs the experimental kernel
but is otherwise expendable.

This document describes both single machine and dual machine arrangements. In the text below the
\newterm{target system} is the system where you will ultimately run the experimental kernel. The
\newterm{development system} is the system where you will do your development work. The
\newterm{experimental system} is the system where you will do testing. In a single machine
configuration the target system and the development system are the same, and there is no
experimental system. In a dual machine configuration the target system and the experimental
system are the same.

Additional details about how to set up and use the dual machine configuration as two connected
virtual machines are in the companion document \textit{DevBox and HackBox} in the same location
as where you found this document.

\section{Compiling the Kernel}

This section describes how to compile the Linux kernel. It is intended to support individuals
compiling the kernel for the first time. Note that this section assumes you are using a recent
(6.x) version of the kernel.

\subsection{Downloading and Unpacking}

Before compiling the kernel you will need to obtain a copy of its source code. Your Linux
distribution should come with the source code of the kernel (it's required by the GPL). However,
because the kernel source is large and because most users do not require it, the kernel source
is not normally installed by default. Furthermore, each distribution tends to specialize the
kernel in various ways. To build a new kernel from a particular distribution's kernel source
package, you should consult the documentation for that distribution.

If you are interested in kernel development, or if you want to always use the latest kernel, I
recommend that you download the stock kernel source from \url{http://www.kernel.org/}. This site
is the official repository for Linux kernels. The kernels there are generic in the sense that
they haven't been customized for any particular Linux distribution. As a result you may need to
configure the kernel (as described below) in a non-default way before it will boot your system
cleanly.

If you are interested in keeping up with the absolute latest version of the kernel you can check
out its source code from GitHub here \url{https://github.com/torvalds/linux}. This would allow
you to do your development in your own fork and periodically merge upstream changes into your
work. It would also allow you to share your work with a team of collaborators. However, using
Git to do these things is outside the scope of this document.

This document describes the build process specifically for kernel version 6.9.3 on 64~bit Ubuntu
Linux 24.04. The commands below reflect this version number and platform. If you are using a
different version or building on a different platform, modify the commands as appropriate. Note
that you may need to install some extra packages on your system in order to do kernel
development. This document does not describe which packages are necessary nor how to install
them. Consult the documentation for your distribution for more information. If you are missing
required packages, some commands below may not work. That is a sign that additional packages may
be necessary.

It is useful to check which version of the kernel is running on your intended target system. You
can do this with a command such as:

\begin{commands}
  \$ cat /proc/version
\end{commands}

At the time of this writing, the version reported on Ubuntu Linux Server 24.04 is ``Linux
version 6.8.0-35-generic.'' Note that the experimental kernel I am proposing to install (6.9.3)
is somewhat, but not extremely, newer than the one running on the system already. This is a good
situation. The newer kernel will likely support all the features the current system expects,
while not being so new as to introduce incompatibilities. This is particularly important if you
are planning to run the experimental kernel on your development system (i.e., a single-machine
configuration).

The kernel source is kept in a compressed archive called a \newterm{tarball}. For version 6.9.3,
the name of this archive is \filename{linux-6.9.3.tar.xz}. The last component of the version
number is a release level. As 6.9 matures, it advances through several releases.

Once you download the tarball you will need to unpack it somewhere on your development system.

\begin{commands}
  \$ unxz < linux-6.9.3.tar.xz | tar xf -
\end{commands}

Unpacking the tarball will create a \filename{linux-6.9.3} directory beneath the current
directory. While doing kernel development it is probably best to unpack the kernel somewhere in
your home directory. That will make it easy for you to do development under your normal user
account. You will need to be root to install the kernel on your system (e.g., in a
single-machine configuration), but you can configure and build it as an ordinary user.

\subsection{The C Library}

The application interface to the kernel is by way of the C library. In theory, when the kernel
is updated, the C library needs to be rebuilt so that it can take into account any changes in
kernel-specific data structures provided by the new kernel. Applications that link to the C
library statically (that is \emph{not} using the dynamic shared library) would also need to be
recompiled.

In addition, some header files from the kernel source are usually also in
\filename{/usr/include} where applications that need them (for example, applications making
direct system calls) can access them. Again, in theory, when the kernel is updated those header
files also need to be updated as well.

A Linux distribution installs a version of the C library and kernel headers that correspond to
the installed kernel. When you update your kernel using the distribution's normal update system,
these things are also updated if necessary. When you manually update your kernel, however, you
might have to update these things yourself.

That said, changes in the kernel that necessitate rebuilding the C library are relatively rare.
If the kernel you are installing is not too different from the one you are already using, you
can probably get away without these additional complications.

Some distributions use symbolic links from \filename{/usr/include} into the kernel source tree.
In that case, they will install the kernel headers under \filename{/usr/src} for the installed
kernel even if they do not install the full kernel source. In a case like this you can change
the symbolic links in \filename{/usr/include} to point at the new headers; however you should in
theory also rebuild the C library and statically linked applications if you do this.

If this sounds complicated and unreliable you aren't the only person who thinks that. There have
been discussions among the kernel developers about how to ``fix'' this situation, but at the
time of this writing I'm not sure how those discussions have concluded.

% TODO: What is the current (summer 2024) status of this situation? The paragraphs above were
% written quite some time ago.

\subsection{Configuring}
\label{sec:configuring}

Before you compile the kernel you should \newterm{configure} it. This involves selecting which
features you want enabled in your kernel and which features should be compiled as modules that
can be loaded later. The configuration process creates two files: \filename{.config} in the root
of the kernel source tree and \filename{autoconf.h} in
\filename{include/generated}.\footnote{The \filename{autoconf.h} file is created as part of the
build process. It does not exist until you actually compile the kernel for the first time.} The
file \filename{.config} gives the make utility access to your desired configuration. Make uses
this information to control which files are compiled and how. The file \filename{autoconf.h} is
included into the various kernel source files (and also in external modules) and gives the C
compiler access to your desired configuration. Programmers can use \#ifdef/\#endif directives in
the C source to selectively compile different code depending on the configuration options
chosen.

Your Linux distribution has most likely already created a kernel configuration that is
compatible with the software environment of that distribution. For example the start-up scripts
may depend on certain features being enabled in the kernel. Using that configuration as a
starting point allows you to configure your new kernel to match the existing configuration as
closely as possible resulting in a minimum of problems.

To create a configuration based on some currently running kernel, you will need to first obtain
a copy of the configuration file for the that kernel. Depending on how the running kernel was
configured a compressed version of that file might be in \filename{/proc/config.gz}. My Ubuntu
24.04 distribution stores a copy of the current configuration in
\filename{/boot/config-6.8.0-35-generic} where 35 is the current (at the time of this writing)
release number of the Ubuntu-flavored kernel. You can read the file \filename{/proc/version} to
find the exact version of the running kernel. That information might be useful for determining
which configuration file is the most appropriate if there is more than one.

Copy the existing configuration file to the root of your new kernel source tree under the name
\filename{.config}. Then run the command \command{make oldconfig} to update that configuration
for use with the new kernel. For example, if you are building a new kernel for your development
system:
\begin{commands}
  \$ cp /boot/config-6.8.0-35-generic .config \\
  \$ make oldconfig
\end{commands}

If you are building a kernel to run on an experimental system you should create a
\filename{.config} that is suitable for that system. Thus, instead of copying the existing
configuration of the development system, you should, in theory, instead take it from the
experimental system. In practice, if you are using the same Linux distribution on both the
development and experimental system, it is reasonable to use the configuration file from the
development system as the basis for your configuration.

There are two kinds of issues that will be reported by the \command{make oldconfig} command.
First, configuration options in the existing kernel that are not in the new kernel (because they
have been removed or renamed, or because the existing kernel has been extended by the
distribution) produce warnings when they can't be mapped into the new configuration.

Second, configuration options in the new kernel that are not in the existing kernel (because
they are new) will prompt you to choose between `Y' (meaning compile the feature into the
kernel), `N' (meaning don't compile the feature), or in some cases `M' (meaning compile the
feature as a module). You can just hit `enter' to accept the default in most cases.\footnote{You
can also select `?' to get help information about an option.}

Next, you may need to make a manual adjustment to the \filename{.config} file with respect to
the list of trusted keys known to the kernel. Linux has a facility that allows modules to be
digitally signed with signatures that are checked by the kernel when each module is loaded.
During the kernel build process, a public/private key pair is automatically generated by the
kernel for this purpose (although you can provide your own if desired). Internally, the kernel
keeps a list of public keys known to it on a ``key ring.''

In addition to the generated (or provided) public key mentioned above, it is also possible to
provide a list of other public keys as X.509 certificates in a PEM-formatted file. These keys
are built into the kernel image during the build process. For our purposes, we don't need to
provide such a file, but you'll need to adjust the configuration to say so. Search the
\filename{.config} file for the following configuration parameters:

\begin{commands}
  CONFIG\_SYSTEM\_TRUSTED\_KEYS
  CONFIG\_SYSTEM\_REVOCATION\_KEYS
\end{commands}

Set their values to the empty string (there may be non-existent files mentioned by default). The
values for these parameters are leftovers from the kernel configuration you copied and thus
reflect values being used by your existing kernel's distributor.

After the initial configuration process is complete you can use:
\begin{commands}
\$ make menuconfig
\end{commands}
OR
\begin{commands}
\$ make gconfig
\end{commands}

to further refine the configuration. You can also use these commands to change the configuration
later if desired. Note that \command{make menuconfig} requires that you have the curses terminal
handling library available and \command{make gconfig} requires that you be running a graphical
desktop with the appropriate GTK+ graphical libraries available. \todo{Is \texttt{make gconfig}
still even supported? It seems to require an old version of the GTK+ libraries.}

\textit{Warning!} If you modify the configuration with either of the commands above you will
need to do a full kernel rebuild. This takes a long time so if you aren't prepared to do that be
careful not to accidentally change anything when just reading the configuration.

If you are building a kernel for experimentation purposes you may want to enable some debugging
features in the kernel configuration. I invite you to explore the options under the ``Kernel
hacking'' heading. Even if you decide to not activate any debugging features at this time it
would be good for you to be aware of the possibilities in case you want to try them later.

Note that debugging features, and the checks they imply, will impact the performance of your
kernel in terms of both space and time. This is why many of them are off by default. In fact,
some checks are so expensive that I do not recommend using them in a kernel built for general
use (although they may be acceptable on the experimental system). Consult the help information
on each option for more information.

If you wish to debug your kernel using a source level debugger you will want a kernel debugging
option turned on. To find this option, look under ``Kernel hacking'', then under ``Compile-time
checks and compiler options'', and finally under ``Debug information''. Full debugging
information is the default in the Ubuntu 24.04 configuration. However, \emph{be aware that this
option greatly increases the amount of disk space required to build the kernel} since every
object file produced by the compiler contains debugging symbols.\footnote{Several gigabytes of
disk space are required for a full build using debugging.}

This option is appropriate if you are creating a User Mode Linux kernel (see
Section~\ref{sec:UML}) or if you plan to use a kernel debugger (such as KGDB) or a tool to
analyze kernel crash dumps. If you do not plan to use a source level debugger you can save a lot
of disk space by setting the ``Debug information'' option to ``Disable''. 

If you are building a kernel for use on an experimental system, and you wish to debug it
remotely using KGDB, you will want to enable KGDB support in the kernel configuration in the
``Generic Kernel Debugging Instruments'' submenu beneath ``Kernel hacking'' (this is also the
default in the Ubuntu 24.04 configuration). I say more about setting up and using KGDB in the
companion document \textit{DevBox and HackBox}.

\subsection{Building}
\label{sec:building}

To actually build the kernel and all of its supporting modules do:
\begin{commands}
  \$ make
\end{commands}

A kernel build takes a long time. There is a lot of code. Note that this will compile most
drivers (as modules) even though your system will likely never use them. Nevertheless, you may
find some drivers useful, especially in an experimental context, so it doesn't hurt to build
them all.

The build process may require various libraries that you do not have installed on your system
initially. If the build fails, look at the reason and then try to install the necessary
package(s) to satisfy any missing requirements. Run the \command{make} command again to restart
the build. It will pick up where it left off. You may need to do this several times. However,
once you have all the necessary libraries installed, future builds should go more smoothly.

\subsection{Installing}

You do not need to be root to configure and compile the kernel. If you unpack the kernel source
in an area where you have read/write permission, you should be able to build it as an ordinary
user. However, you do need to be root to copy the new kernel to a place where it can be used to
boot a machine.

Once the kernel has been built you should first copy the various compiled modules to the proper
directory under \filename{/lib/modules} so that the running kernel can find them. This is
accomplished by doing:
\begin{commands}
  \# make INSTALL\_MOD\_STRIP=1 modules\_install
\end{commands}

The INSTALL\_MOD\_STRIP option removes debugging information from the modules as they are
installed. This greatly reduces the amount of disk space used by the module library and by the
initial RAM disk (described below). In low memory systems saving memory can be essential since
the full module library, with debugging information included, is very large. You may be
motivated to configure your experimental system with a minimal amount of memory since it won't
be used for any ``real'' work. As a result, without stripping the modules, it is likely that the
initial RAM disk will be too large for a minimally configured system to use.

One disadvantage of stripping debugging information from the modules is, obviously, you won't be
able to step into those modules or set break points in them when debugging the kernel. This
might be an issue if you are trying to use the debugger to study the operation of the kernel.
However, if you are mostly interested in debugging your own kernel modifications or kernel
modules, removing debugging information from the distributed modules will probably not cause you
any issues.

Another aspect of stripping debugging information is the effect it has on module signatures.
Normally, the debugging information is covered by the signature. Thus, the signature is
invalidated or removed (\todo{Which is it?}) if the debugging information is stripped afterward.
However, the \command{make} command above strips debugging information first, before making the
signatures, so you end up with signed, stripped modules as desired. \todo{The \command{file}
  command uses the phrase ``with debug\_info'' as opposed to ``stripped.'' In fact, it seems to
  use ``stripped'' for something else. The terminology here should be made consistent.}

You should install modules even on your development system because the module library of a
kernel is used during the compilation of external modules for that kernel. However, it is safe
to install modules for a kernel even if you usually run a different kernel. Each kernel has its
own private directory under \filename{/lib/modules}.

If plan to run your new kernel on an experimental system (called "hackbox" in the commands
below) you should also copy the modules to that system where they can be used. On the
development system, after installing modules, do the following:
\begin{commands}
  \# cd /lib/modules \\
  \# tar cf - 6.9.3 | gzip > modules-6.9.3.tar.gz \\
  \# scp modules-6.9.3.tar.gz hackbox:/lib/modules
\end{commands}

Unpack it on the experimental system using:
\begin{commands}
  \# cd /lib/modules \\
  \# gunzip < modules-6.9.3.tar.gz | tar xf -
\end{commands}

You should next copy and rename three files from your freshly built kernel to the
\filename{/boot} directory on the experimental system. For example, you might do the following:
\begin{commands}
  \# cd /home/student/linux-6.9.3 \\
  \# scp arch/x86/boot/bzImage hackbox:/boot/vmlinuz-6.9.3 \\
  \# scp .config hackbox:/boot/config-6.9.3 \\
  \# scp System.map hackbox:/boot/System.map-6.9.3
\end{commands}

The \filename{System.map} file contains a list of all symbols in the kernel and their
corresponding addresses. This can be useful for debugging and for interpreting stack traces in
kernel oops messages. \todo{Why is it important for this file to be in \filename{/boot}?}

Finally, I recommend using \command{ls -l} in the \filename{/boot} directory of the experimental
system to check file ownership and permissions. Use the \command{chown} and \command{chmod}
commands as appropriate to match the owners and permissions on the new files to those of the
existing files. While this is not an essential step, it gives the installation a clean look, and
it may have importance from a security and system maintenance point of view.


\subsubsection{Making \filename{initrd}}

Because modern Linux systems are so highly modularized it is possible that the kernel will need
to load a module in order to read the file system. This creates a problem: how can it load a
file system support module from the file system? To get around this, Linux boots in two phases.
During the first phase, the bootloader loads a pre-defined RAM disk image into memory along with
the kernel. The kernel uses this RAM disk image as it's initial root file system. Certain
programs and kernel modules can be loaded out of this RAM disk image. Once that is done, the
root file system is changed to be the normal hard disk and the usual start-up scripts are
executed.

Manually creating this initial RAM disk is a somewhat involved procedure. Fortunately there is
utility program named \command{mkinitramfs} that does most of the work. On a Ubuntu system the
command is:
\begin{commands}
  \# mkinitramfs -o /boot/initrd.img-6.9.3 6.9.3
\end{commands}

This creates a RAM disk using the same modules as in the existing configuration, except that it
will use the modules for the right kernel version. If you attempt to use the old RAM disk, it
will contain modules for the old kernel which won't load into the new kernel.

You should run this command on the experimental system where you plan to run the new kernel. Be
sure you have the module library installed on that system, and be sure the new kernel
configuration file is also installed in \filename{/boot}. The \command{mkinitramfs} command
makes use of both of those resources.

\subsubsection{Configuring GRUB}

Once you have \filename{vmlinuz-6.9.3} and \filename{initrd.img-6.9.3} in the \filename{/boot}
directory of your target system you only need to update your bootloader to provide an option to
boot the new kernel. This can be done by cloning the information for the existing kernel to a
new menu entry and modify the names of the kernel image file and RAM disk file. The precise
steps for doing this will depend on the bootloader you use.

On a Ubuntu system this is easily accomplished by using the \command{update-grub} command. This
command searches \filename{/boot} for kernels installed there and composes a suitable GRUB menu
for them.

The next time you boot your system if you press the left-hand shift key early in the boot
process you will see the GRUB boot menu. From there you can select your new kernel. \todo{Say
more about setting up GRUB options.} However, if the experimental kernel is the newest kernel on
the system, it will be booted by default.

\subsubsection{Installing on Floppy}

\textit{This section is very old and needs to be rewritten (or removed?). For one thing it needs
  to explain how to handle the initial RAM disk. For a second thing it should probably really
  talk about setting up a flash drive instead of a floppy (who has floppy drives?).}

The following instructions pertain to users who are booting Linux from a floppy disk. Note that
this is not the normal configuration (although it is sometimes useful in lab situations).

\begin{enumerate}

\item Make a copy of your boot floppy. Never overwrite a known working kernel with one that you
  just compiled! First make a backup of the working kernel and be sure that you can boot the
  working kernel if necessary.

  On Windows you can back up your boot floppy with the \command{diskcopy} command. Open a
  Windows command prompt and do:
  \begin{commands}
    C:$\backslash$> diskcopy a: a:
  \end{commands}

  You will be prompted for the source disk and then for the target disk. Note that you can't
  just copy the files from one disk to another! A boot floppy contains special information in
  the boot sector that will not be copied by the usual file copying operations.

  On Linux you can use the \command{dd} command to copy disks raw. Insert the source floppy and
  do:
  \begin{commands}
    \# dd if=/dev/fd0 of=/tmp/floppy.img bs=1024
  \end{commands}

  Then insert the target floppy and do:
  \begin{commands}
    \# dd if=/tmp/floppy.img of=/dev/fd0 bs=1024
  \end{commands}

  You can remove the temporary file afterward if you wish. See the manual page for the
  \command{dd} command for more information.

\item Next insert the boot floppy where you want the new kernel to go and mount it. This can be
  done with a command such as:
  \begin{commands}
    \# mount /dev/fd0 /mnt/floppy
  \end{commands}

  Use whatever directory is most appropriate if \filename{/mnt/floppy} is not available (there
  should be a \filename{/mnt} directory at least).

\item Copy \filename{arch/x86/boot/bzImage} to \filename{/mnt/floppy}, renaming it to
  \filename{vmlinuz} in the process. This will overwrite the \filename{vmlinuz} on the floppy
  with the new kernel.

  There are some control files on the floppy as well that you could edit. However, if you use
  the same name (and you might as well since the floppy isn't big enough to hold both kernel
  images) the control files should already be configured properly.

\item \emph{Very Important!} Unmount the floppy before physically removing it. This is necessary
  because Linux keeps disk blocks in its cache even for floppy disks. This means that the entire
  file isn't actually put on the floppy until you unmount it.
  \begin{commands}
    \# umount /mnt/floppy
  \end{commands}

\end{enumerate}

Now you can reboot the machine from your new boot floppy to check your new kernel.

\section{User Mode Linux}
\label{sec:UML}

\textit{This section is old and needs to be reviewed and updated. It has been a while since I
  have built a User Mode Linux kernel.}

Setting up a completely independent experimental system is a nice way to do kernel development.
However, there are times when it may not be desirable. For example, you might want to do kernel
development on a machine that you depend on for your normal work and yet not want to risk
running an experimental kernel directly on that system. One option is to use virtualization
software to create a virtual machine for a separate experimental system (as described in
\textit{DevBox and HackBox}). However, if your primary machine is also running Linux, another
approach is using User Mode Linux (UML).

The Linux kernel is cross-platform and with suitable cross-compilers can be compiled on one
platform for execution on another. User Mode Linux is treated as a special ``platform.''
However, the UML kernel runs on top of a host Linux system as an ordinary process. All access to
hardware is translated into system calls made against the host system. UML thus allows you to
run a custom Linux kernel alongside your regular applications. You don't even need to be root on
the host system.

Another advantage to UML from a kernel development point of view is that it allows you to debug
the kernel using an ordinary source level debugger such as \command{gdb} without the
complexities of setting up a separate machine and remote debugging. I will discuss how to do
this in more detail later in this document.

\subsection{Compiling UML}

The procedure for compiling the User Mode Linux kernel itself is simple. In what follows I will
assume you are using a 4.x kernel. The 4.x kernel comes with UML as one of the officially
supported architectures. First, unpack the kernel source code to a suitable working directory.
Next run the command:
\begin{commands}
  \$ make defconfig ARCH=um
\end{commands}

It is important to use the default configuration generated by \command{defconfig} as the
starting point for your kernel configuration. \emph{Do not try to use the configuration of the
  running kernel.} It is also important to add the \command{ARCH=um} option to the command line.
This tells the build system that you are cross compiling to a different architecture.

Next run either:
\begin{commands}
  \$ make menuconfig ARCH=um
\end{commands}
OR
\begin{commands}
  \$ make gconfig ARCH=um
\end{commands}

It is important to consistently use the architecture specifier.

Under ``UML Specific Options'' be sure that ``Host filesystem'' is selected. This will allow the
UML kernel to access the file system of the host; an easy way to share files between the host
and a running UML system. Under the ``Kernel hacking'' option (on the top level menu) be sure
the ``Compile the kernel with debug info'' and the ``Compile the kernel with frame pointers''
options are both selected. These options make it possible to properly debug the UML kernel with
\command{gdb}. You may or may not want to set some other debugging related options. Save these
changes.

You are now ready to build the kernel using the command:
\begin{commands}
  \$ make ARCH=um
\end{commands}

When the build is complete you will have an ordinary executable file named \command{linux} in
the root directory of the source tree. Before you can use it, you will need a root file system.

\subsection{UML Root File System}

User Mode Linux reads its root file system out of a file in the host's file system. This file
must be set up so that it contains all the normal programs and tools Linux needs to boot. You
may also want development tools or other programs inside your UML's root file system. Although
you may be able to download a root file system, matching the root file system with the precise
kernel version and options you want to use can be tricky. The reason for this is that during the
boot process, typical Linux configurations read various modules out of the root file system to
enable support for features needed by the startup scripts. These modules must be compatible with
the running kernel. If you are using a kernel version that is not compatible, the boot process
is likely to fail, or at the very least report many errors.

Sometimes you can download a UML kernel along with a matching root file system. However, as a
kernel developer, the kernel you want to use is not arbitrary; it is a specially configured
kernel of your choosing. As a consequence of this, the ideal path is to build your own custom
root file system for kernel development. The procedure is as follows.

\todo{FINISH ME! The description below is very incomplete.}

\begin{enumerate}

\item Create a file to hold the root file system. This file may need to be rather large,
  depending on how much material you plan to install into the UML environment. Use a command
  such as:
  \begin{commands}
    \$ dd if=/dev/zero of=root\_fs bs=1M count=512
  \end{commands}

  This command creates a file named \filename{root\_fs} with a size of 512 megabytes. This file
  is initially all zeros. It will be treated as raw disk image.

\item The image created above must then be formatted with a suitable file system.
  \begin{commands}
    \$ mk2efs -j root\_fs
  \end{commands}

  This command creates an ext3 file system (which is the same as an ext2 file system with a
  journal created by the \command{-j} option) inside the \filename{root\_fs} file. You may need
  to specify the path to the \command{mk2efs} program; it is typically not in the path of
  ordinary users. The use of ext3, or some other journaled file system, is recommended. Since
  this environment will be used for kernel development, kernel crashes are likely, and it is nice
  to have the extra safety inherent in using a journaled file system.

\item Mount the root file system so that you can access its contents. To do this you will need
  to be the root user. First create a suitable mount point. I suggest an empty directory named
  \filename{root}. Then issue:
  \begin{commands}
    \$ mount -o loop root\_fs root
  \end{commands}

  This command uses the loop back driver (which must be supported in your host kernel) to mount
  the file system contained in \filename{root\_fs} onto the mount point \filename{root}.

\item You will now find an empty partition beneath the mount point, ready for you to set up your
  root file system. Configuring a root file system for use with Linux is a non-trivial exercise.
  Lack of space in this document prevents me from going into the details here. Please refer to
  other documentation for more information.

\item Once the root file system is ready you should (as the root user) unmount it before you try
  to use it with UML.
  \begin{commands}
    \$ umount root
  \end{commands}

  You might want to make a backup copy, perhaps in compressed form, of your fresh root file
  system in case you destroy your working copy while setting up UML or doing kernel development.

\end{enumerate}

\subsection{Running UML}

Make sure the root file system is named \filename{root\_fs}. Execute \command{linux} to boot
User Mode Linux.\footnote{You can name the root file system something else, in which case you
  need to add the \texttt{ubd0=} boot option to the command line to specify it.} Log in as the
user root. Typically, depending on the root file system you are using, the password will either
be blank or also root.

Once you have logged in you can use the command:
\begin{commands}
  \# mount none /mnt -t hostfs
\end{commands}
to mount the root of the host file system onto the \filename{/mnt} directory inside the UML
environment. This allows you to copy files to and from the host file system. You may now use
User Mode Linux in a manner very much the same as any other Linux system.

\todo{Talk about setting up disk partitions under UML.}

To debug the running UML process, open another window on the host machine. Use a command such
as:
\begin{commands}
  \$ ps aux | grep linux
\end{commands}
to search for information about the running UML process. You will find several entries because
UML is a multithreaded application. Note the process ID of the first entry. Then launch the
\command{gdb} debugger, attaching \command{gdb} to the process of interest. For example:
\begin{commands}
  \$ gdb linux 1234
\end{commands}
where 1234 is the process ID of the running UML system.

Once \command{gdb} has started you will want to execute the command \command{handle SIGUSR1
  nostop noprint}. I'm not entirely sure what this command does exactly, but it has something to
do with the way \command{gdb} deals with multiple threads. In particular, without this command
\command{gdb} will stop the UML system frequently because of SIGUSR1 signals.

Notice that when \command{gdb} attaches to a process, that process is stopped. Your UML session
will appear dead. However, by issuing the \command{continue} command to \command{gdb} you can
cause your UML session to resume normally. Use \^C in the \command{gdb} window to interrupt the
UML session at any time. You can now set break points and single step the Linux kernel as you
might any other process.

\todo{Talk about how to set a break point inside a module.}

\section{Code Browsing Tools}
\label{sec:code-browsing}

The Linux kernel is very large and finding one's way around in it can be a major chore. To
simplify the navigation of large programs there exists a number of code browsing tools. I
recommend using one or more of these tools when working with the Linux kernel source. In this
section I talk about how to configure a few of these tools for use with the Linux kernel.

Note that you should only set up code browsing tools \emph{after} you've applied any patches to
the source code. Patches will, of course, modify files and change the line numbers where
functions are defined, etc. If you index the source and then apply patches or make other changes
you will want to reindex the source afterward. However, it shouldn't matter if you've compiled
the kernel first or not. Code browsing tools are normally smart enough to ignore object files.

\subsection{Cscope}

The \command{cscope} tool is a simple but effective code browser with a long history. It reads a
collection of C files and builds an indexed database that can be used to quickly look up
declarations and points-of-use for any symbol.

The script below launches \command{cscope} on the Linux kernel. You should edit the three
variable definitions at the top of the script to suite your system. The script does not index
the entire kernel code base. In particular, it skips the driver hierarchy and only indexes the
x86 architecturally specific code. This makes the database size more manageable and reduces the
number of duplicate declarations the tool returns.

\begin{verbatim}
#!/bin/bash

# Set a few variables. Change here for your system.
CSCOPE_DIR=/home/student/cscope
CSCOPE_FILE=$CSCOPE_DIR/cscope.files
LNX=/home/student/linux-6.9.3

# If the database hasn't yet been created, then create it.
if [ ! -f $CSCOPE_FILE ]; then

  # Build file list. Exclude uninteresting regions.
  cd ~
  find $LNX \
    -path "$LNX/arch/*" ! -path "$LNX/arch/x86*" -prune -o \
    -path "$LNX/Documentation*" -prune -o \
    -path "$LNX/scripts*" -prune -o \
    -path "$LNX/tools*" -prune -o \
    -path "$LNX/drivers*" -prune -o \
    -name "*.[chxsS]" -print > $CSCOPE_FILE

  echo Creating database...
  cd $CSCOPE_DIR
  cscope -b -q -k

  echo Creating tags...
  ctags -L $CSCOPE_FILE
  if [ -f tags ]; then
    mv tags "$LNX"
  fi

fi

# Run cscope
cd $CSCOPE_DIR
cscope -d
\end{verbatim}

In addition to using a dedicated code browsing tool, many programmer's text editors have a
feature that allows them to read a ``tags'' file containing cross-reference information about
entities declared and defined in a program. The script above uses the \command{ctags} command to
create such a file for the Vim editor.

It is natural to create a tags file for your editor at the same time as you create the
\command{cscope} database. This is because \command{cscope} will launch your editor (Vim is the
default) whenever you ask it to display a file. Once in the editor it is convenient to continue
your browsing experience using editor tags commands.

If you are an \command{emacs} user instead, you can create a tags file for that editor with the
\command{etags} command. Set the EDITOR environment variable to \command{emacs} to override
\command{cscope}'s default editor setting.

\todo{FINISH ME! Need to talk about how to use \command{cscope} and the tags file.}

\end{document}
