%%%%%%%%%%%%%%%%%%%%%%%%%%%%%%%%%%%%%%%%%%%%%%%%%%%%%%%%%%%%%%%%%%%%%%%%%%%
% FILE    : DevBox-HackBox.tex
% AUTHOR  : (C) Copyright 2024 by Peter Chapin
% SUBJECT : DevBox and HackBox installation and use.
%
% DevBox and HackBox are virtual machine images for use during the development of GenericFS and
% for operating system hacking in general. They were developed originally for my operating
% systems class at Vermont State University.
%
%       Peter Chapin
%       Computer Information Systems
%       Vermont State University
%       141 Lawrence Pl.
%       Williston, VT. 05495
%       peter.chapin@vermontstate.edu
%%%%%%%%%%%%%%%%%%%%%%%%%%%%%%%%%%%%%%%%%%%%%%%%%%%%%%%%%%%%%%%%%%%%%%%%%%%

% ++++++++++++++++++++++++++++++++
% Preamble and global declarations
% ++++++++++++++++++++++++++++++++
\documentclass{article}

\usepackage{fancyvrb}
\usepackage{hyperref}
\usepackage{url}

% \pagestyle{headings}
\setlength{\parindent}{0em}
\setlength{\parskip}{1.75ex plus0.5ex minus0.5ex}

% ------------
% New Commands
% ------------

% Add commands in alphabetical order.
\newcommand{\command}[1]{\texttt{#1}}
\newcommand{\filename}[1]{\texttt{#1}}
\newcommand{\newterm}[1]{\textit{#1}}
\newcommand{\todo}[1]{\textit{TODO: #1}}

% ----------------
% New Environments
% ----------------

% An environment to display a sequence of commands.
\newenvironment{commands}
  {\begin{quote} \tt}
  {\end{quote}}

% +++++++++++++++++++
% The document itself
% +++++++++++++++++++
\begin{document}

% ----------------------
% Title page information
% ----------------------
\title{DevBox and HackBox}
\author{\copyright\ Copyright 2024 by Peter Chapin}
\date{Last Revised: August 14, 2024}
\maketitle

\tableofcontents

\section{Introduction}

When doing operating system work, you ideally want to run your experimental software on a
different computer than the one you use for development. An error in an operating system module
or driver has the potential to corrupt the entire machine; testing on your development system
may lead to headaches if such corruption occurs. Also, it is generally not possible to debug an
operating system while running the debugger tool on the same machine. Two machines are needed so
the debugger and the debugee can be properly separated.

In this document I call the development system ``DevBox.'' It is a full environment with all the
usual tools and conveniences. It is on DevBox where you spend most of your time working and
compiling your programs. It is on DevBox where you run source level debuggers.

The experimental system I call ``HackBox.'' It is a simplified environment that is minimally
configured. The smaller, lighter configuration makes testing and error recovery easier.

This document describes how to set up the preconfigured DevBox and HackBox virtual machines
provided for the operating systems course at Vermont State University. I describe how to install
the VirtualBox VM software, how to import and run the DevBox and HackBox virtual machines, and
how to exercise the system to verify that it is set up properly. At the end of this document I
include some notes on how DevBox and HackBox were built. That information is for reference only
and not essential for using the systems, but may be of interest to anyone who wants to build (or
rebuild) the virtual machines.

\section{Installing VirtualBox}

The CIS lab machines on the Williston (BLP-210) and Randolph (CON-106) campuses should already
have the VirtualBox software installed. If you are using a lab machine you can skip this section
and continue with Section~\ref{sec:booting-guests}. If you intend to run DevBox and HackBox on
your personal machine you will need to first install VirtualBox on your system.

\textit{Important note for macOS and Windows-for-ARM users!} The DevBox and HackBox virtual
machines provided are x86\_64 guests. At the time of this writing, it is not easy to run these
guests on Mac computers based on the Apple ARM processors (Apple Silicon: M1, M2, etc.) or on
the new Windows systems based on the Qualcomm Snapdragon processors. However, on macOS at least,
you can try installing the UTM virtualization app from the App Store and then running DevBox and
HackBox in emulation mode rather than in virtualization mode\footnote{\todo{Add more details
about how to import the virtual machines into UTM.}}. The performance of the guest systems will
be significantly less, but they may still work adequately. Alternatively, you could either find
an x86\_64 system on which to run the virtual machines, or you can use one of the lab machines.

DevBox and HackBox together consume significant resources. Before trying to install them on your
personal machine you should be sure you have at least 4 GiB of memory\footnote{8 GiB or more is
preferred.} and 40 GiB of free disk space. You also need to have hardware virtualization support
turned on in the BIOS of your host computer. Most machines come with this feature turned on by
default, so if you are unsure, you can probably ignore this issue until you have an actual
problem. Older machines may require you to enter the BIOS and activate this feature. If your
machine is extremely old it may not support hardware virtualization at all. Finally, DevBox and
HackBox assume that your host has at least two cores.

If the host does not meet the requirements above it may be possible to reconfigure the virtual
machines so that they will work anyway. However, you may have to sacrifice some features or
endure suboptimal performance in that case.

If you are using a Windows machine with the Hyper-V service running, you \emph{may} run into
issues running VirtualBox\footnote{This tends to be less of a problem with recent versions of
VirtualBox.}. The Hyper-V service is a ``type 1'' hypervisor whereas VirtualBox is a ``type 2''
hypervisor. These two types fundamentally conflict with each other. In fact, running VirtualBox
on system with Hyper-V enabled didn't work at all until relatively recently. Modern VirtualBox
versions \emph{do} work on top of Hyper-V, but performance can be very slow under some
conditions. When you launch a guest VM, look in the lower right corner of the window. If you see
a blue-gray box with a ``V'' then you are running normally. If you see a green turtle, you
probably have Hyper-V enabled. You could disable Hyper-V to get better performance, but be aware
that this will disable any other virtualization system you may have installed that relies on it
(i.e., docker or WSL). My recommendation is that you don't worry about this unless you observe
actual performance issues.

Begin by downloading VirtualBox from \url{http://www.virtualbox.org}. Be sure to download both
the main installer for your system \emph{and} the Extension Pack. The Extension Pack adds
important functionality that DevBox and HackBox requires. \emph{The virtual machines will likely
not boot in their default configuration without the Extension Pack installed!} The current
versions of DevBox and HackBox have been tested using VirtualBox version 7.0.20. They will
probably work with other versions of VirtualBox that are close to that version.

Run the VirtualBox installer. After the installer completes double click on the Extension Pack
to install that as well.

\section{Booting the Guests}
\label{sec:booting-guests}

DevBox and HackBox are distributed as virtual appliances (OVA for ``Open Virtual Appliance'').
This is a standard file format that includes not only the virtual hard disk (compressed) but
also the virtual machine's configuration. Once you import the virtual appliance into VirtualBox
you can boot a virtual machine by just starting it as you might start a real computer.

First, some terminology. The ``guest'' system is the system running inside the virtual machine.
The ``host'' system is the system running the virtualization software. These terms are widely
used in the virtualization community, and I will use them here.

The virtual appliance format is accepted by several virtualization products. You may be able to
run DevBox and HackBox using some other virtualization product such as VMware. However, that is
an untested configuration.

To import the virtual appliances into VirtualBox follow the steps below.

\begin{enumerate}

\item Download the file \texttt{DevBox-YYYY-MM-DD.ova} and similarly for HackBox. The names of
  the files contain the date when that version was released. The files also have an associated
  MD5 checksum that can be used to check for download errors. It is recommended that, if
  possible, you verify the checksums. Because of the large sizes of the files, errors during the
  download are more likely than usual.

\item Start VirtualBox and select ``Import Appliance'' from the File menu. Follow the prompts.
  This will unpack the OVA file and add the virtual machine to VirtualBox's start menu. Repeat
  this for both virtual appliance files.

\end{enumerate}

In principle no further configuration is necessary since the configurations of the virtual
machines are contained in their original OVA files. However, you might review the machine
settings and tweak them if desired. Note especially the amount of memory allocated to the
virtual machines. \emph{DevBox is configured to use 1.5 GiB of memory and HackBox is configured
to use 0.5 GiB of memory. If your host computer has less than 4 GiB of memory you may want to
adjust the configured values downward.} If your host computer is well-endowed with memory you
might consider increasing the amount available to the virtual machines, particularly DevBox. A
good rule of thumb is to allocate no more than 50\% of your system's physical memory to
\emph{all} virtual machines that you intend to run at the same time.

After you have imported the appliances you can delete the OVA files to save disk space. However,
if you have sufficient disk space you might consider keeping the files in case you need to
reinitialize DevBox or HackBox from scratch. Having the OVA files on hand will save you another
long download.

It is perfectly reasonable to boot both DevBox and HackBox at the same time; in fact this is
often necessary for the kind of work you'll be doing. Each virtual machine runs in its own
window. Be aware that you should boot DevBox before booting HackBox. The two virtual machines
share a named pipe that DevBox creates. If you boot HackBox first, you will see an error about
not being able to connect to the (non-existent) pipe. This is mostly just an annoyance. It only
matters if you are trying to do kernel debugging on HackBox.

Both DevBox and HackBox have console-only interfaces. HackBox lacks a graphical interface to
simplify its configuration. DevBox, where you will do most of your work, could benefit from a
graphical interface in principle. However, in an effort to reduce the resources required to run
these VMs, and to minimize problems with graphics driver incompatibilities, DevBox also lacks a
graphical interface. Tasks that require a GUI, such as browsing the web, can be done on the host
system.

The Linux distribution running inside both virtual machines is Ubuntu 24.04, 64-bit. Both VMs
are running the server version of Ubuntu since that version is more resource-friendly.

\section{Logging Into the Guests}
\label{sec:logging-in}

The configuration of DevBox and HackBox include a second network adapter that connects to the
host-only network (normally named ``VirtualBox Host-Only Ethernet Adapter'' in VirtualBox's
network manager tool). This network is managed entirely by VirtualBox and is not connected to
the host's physical network interface. It is used to allow DevBox and HackBox to communicate
with each other and with the host.

By default, the host-only network is configured (by VirtualBox) to use the IP address range
192.168.56.0/24, with the .1 address being attached to a virtual interface in the host. DevBox
is configured to use the address 192.168.56.2 on its second Ethernet interface. HackBox is
configured to use the address 192.168.56.3 on its second Ethernet interface. The first Ethernet
interface on both virtual machines is connected to the NAT network managed by VirtualBox. This
allows both virtual machines to access the Internet for updates and other purposes.

Both DevBox and HackBox are running the OpenSSH server. This allows you to log in to the virtual
machines from the host system using SSH. It is recommended that you use SSH in this way rather
than using the console. There are several reasons for this.

\begin{enumerate}
  \item The console provided by VirtualBox is very limited. It is difficult to scroll back to
    see previous output. It is difficult to copy and paste text. It is difficult to resize the
    window. It is difficult to change fonts, colors, and other styling. The console is useful
    for seeing the boot process and for interacting with the system when the network is not
    available, but it is not a good general-purpose interface.

  \item You probably already have an SSH client installed that you like and are familiar with
    using.
    
  \item Many SSH clients have built-in support for file transfers. This can be very useful for
    moving files between the host and the virtual machines. Even if your SSH client doesn't
    support file transfers, you can use other tools such as WinSCP or \texttt{scp} (e.g., from
    Cygwin or macOS) to transfer files.
\end{enumerate}

A facility is configured that allows DevBox and HackBox to communicate with each other via a
serial connection (in addition to the host-only network). The serial connection is needed for
debugging the kernel on HackBox since when the kernel is being debugged the network will be
non-functional. Thus, both virtual machines have a virtual serial port configured using a named
pipe for inter-machine communication. The format of this name depends on your host OS (the named
pipe is managed by the host). When you try to boot the machines for the first time you might see
an error about being unable to create the necessary named pipe if the name format is incorrect
for your host. On Linux and macOS systems use a name such as \filename{/tmp/hackserial}. On
Windows systems use a name such as
\filename{$\backslash\backslash$.$\backslash$pipe$\backslash$hackserial}. See
\url{https://www.virtualbox.org/manual/ch03.html#serialports} for more information.

Note that the serial port configuration on DevBox is in ``server mode'' which means DevBox is
responsible for creating the named pipe so HackBox can use it. The consequence of this is that
you will need to boot DevBox before you boot HackBox in cases where you want to use both.

Both DevBox and HackBox has a user account named ``student'' with a password of ``frenchfry.''
You should log in as this user. The student user can use \texttt{sudo} when necessary to perform
administrative tasks.

\section{Snapshots}
\label{sec:snapsots}

One major benefit to doing your development inside a virtual machine is that you can use
VirtualBox to \emph{snapshot} your system just before attempting any kind of dangerous or
complicated operation. When you create a snapshot, VirtualBox remembers the entire state of the
system. \emph{Any} change made after the snapshot is provisional. If the system becomes corrupt,
you can just restore to the snapshot and undo all changes made since the snapshot was taken.

The undoing of changes is complete. The process does not rely on the correct operation of the
guest system. Even if the data on the (virtual) hard disk is totally shredded, restoring to a
snapshot will reset every detail of the system back to the state it had when the snapshot was
taken.

With this protection in place you are free to experiment without concern of causing irreparable
damage. For example, if you want to try building and installing a new version of the C
library\ldots\ go ahead! Take a snapshot first, and if the result is a major disaster you can
just roll back to where you started and try again. In the worst case scenario you could delete
your virtual machine and re-import it from the original OVA file. Of course this rolls back all
changes you ever made to the system, but the base configuration will be fully restored.

It is likely you will take snapshots of HackBox frequently during your work. This is because
HackBox will be running experimental kernels and kernel modules and is thus subject to random,
spectacular failures. In contrast, DevBox should remain fairly stable since it only runs
software blessed by Ubuntu and well established third party products. Because of HackBox's
minimal configuration, snapshots of HackBox should be small and quick to make. This is a nice
side effect of using two machines in this way.

\section{Basic Testing}

It is nice after installing DevBox and HackBox to do some simple operations to verify that they
are working for you in a useful way. Keep in mind that none of the steps described in this
section are necessary for DevBox or HackBox to work. They are only intended to give you an
opportunity to exercise the two systems.

\begin{enumerate}

\item Boot DevBox. Use an SSH client of your choice on your host to connect to DevBox at address
192.168.56.2. Log in as the user student.

\item Boot HackBox. Use an SSH client of your choice on your host to connect to HackBox at
address 192.168.56.3. Log in as the user student.

\item Check that you have network connectivity between the two systems. For example, on DevBox
  issue the command:
\begin{Verbatim}
$ ping hackbox
\end{Verbatim}
  Run the corresponding command on HackBox to ping DevBox. This ensures the two systems can
  communicate with each other over the host-only network.

  Both systems also have a second network interface defined that connects via a network address
  translator (NAT) to the host's physical network interface. This gives both machines access to
  the Internet.

\item On DevBox issue the command:
\begin{Verbatim}
$ ssh hackbox
\end{Verbatim}
  This should connect to the ssh server on HackBox over the host-only network and allow you to
  log in (again) as student. You can use the \texttt{scp} command to transfer files between the
  two systems.

  Once you've demonstrated that ssh is working you can log out.

\item \textit{Optional!} On DevBox issue the command:
\begin{Verbatim}
$ minicom
\end{Verbatim}
  Both HackBox and DevBox also have virtual serial ports defined. Those ports
  (\texttt{/dev/ttyS0} in both cases) have been connected together by VirtualBox. HackBox is
  running an \texttt{agetty} process on its ttyS0 serial port to support logins.

  Minicom is a simple terminal program for Linux that can be used to interact with serial ports.
  After starting Minicom you should be able to hit Enter and see a login prompt from HackBox.
  Try logging in to verify that this works.

  When you are done experimenting with this, log out of HackBox and then type CTRL+A followed by
  X in Minicom to exit Minicom.

\item Check that you are using the correct kernel on HackBox by issuing the command (on
  HackBox's console):
\begin{Verbatim}
$ cat /proc/version
\end{Verbatim}
  This command reads a file in the ``magic'' proc file system that, in this case, reports
  information about the kernel version you are using. It should say 6.9.3. This is the
  experimental kernel we will be working with.

\item On DevBox browse around in the directory hierarchy
\begin{Verbatim}
/home/student/linux-6.9.3
\end{Verbatim}
  This is the source code for the experimental kernel. You will become familiar with its
  organization in the future, but a quick look now is a good first step toward that goal.

\item On DevBox change to
\begin{Verbatim}
/home/student/linux-6.9.3
\end{Verbatim}
  and issue the command \command{make menuconfig} to view the kernel configuration menu. The
  configuration you are looking at is the configuration used the last time the kernel was
  compiled. It represents the current configuration of the experimental kernel running on
  HackBox.\footnote{For full details about how to compile Linux, see my companion document
  \textit{Compiling Linux} in the same location where you found this document.} Look around to
  see what kinds of options are available. If you change anything, do \emph{not} save your
  changes. If you save any changes you will end up doing a full kernel rebuild the next time you
  try to compile it. That takes a very long time.

  Once you exit the configuration menu, look at the file \filename{.config} and the file
\begin{Verbatim}
include/generated/autoconf.h
\end{Verbatim}
  to see the results of the configuration process (these files were created previously when the
  experimental kernel was prepared for HackBox). The \filename{.config} file is used by the kernel
  build system to control which source files need to be compiled and how. The
  \filename{autoconf.h} header is included into source files that need to distinguish between
  various configuration options.
\end{enumerate}

\section{Kernel Modification}

In this section you will make a trivial modification to the kernel by editing the kernel source
and rebuilding the kernel on DevBox, transferring the new kernel to HackBox, and then rebooting
HackBox to test your change. You will also back out these changes so that when you are done with
this section there will be no lingering effect on either system.

\begin{enumerate}

\item In the VirtualBox console with HackBox in an ``Off'' state, select the HackBox virtual
  machine and take a snapshot of its current state.

\item On DevBox edit the file \filename{init/version.c}\footnote{Relative paths are relative to
    the root of the kernel source tree unless context indicates otherwise.} and change the value
    of \texttt{linux\_proc\_banner} to include some distinctive text. For example, you might add
    something like ``experimental'' to the existing banner. This will change the contents of
    \filename{/proc/version} once the modified kernel is running. I recommend that you first
    create a backup copy of the original file using a command such as:
\begin{verbatim}
$ cp version.c version.c.orig
\end{verbatim}

  This makes it easy to restore the original file later.

\item Issue the command \command{make} at the root of the kernel source tree to build a new
  kernel reflecting your changes. This should not take too long because the kernel has already
  been built and \command{make} should realize that most of the object files are up-to-date
  (however, it can still take several minutes to do this). If \command{make} appears to be
  recompiling everything something is wrong (possibly you accidentally saved changes when
  looking at the configuration earlier). In that case just cancel the build and don't worry
  about this step to save time. You will need to suffer the full build eventually, however, but
  it might take hours depending on your hardware.

\item Using \command{scp}, transfer the new image to HackBox:
\begin{Verbatim}
$ scp arch/x86/boot/bzImage root@hackbox:/boot/vmlinuz-6.9.3
\end{Verbatim}
  This replaces the boot experimental kernel on HackBox with the version you just compiled
  containing your changed banner. Its good practice to make a backup copy of
  \filename{vmlinuz-6.9.3} first.

\item Reboot HackBox into the experimental kernel and verify that \filename{/proc/version} has
  changed as expected. You do not need to update the \command{grub} configuration in this case
  since you are overwriting the experimental kernel with a new version. You can use the command
  on HackBox to reboot HackBox immediately:
\begin{Verbatim}
$ sudo shutdown -r now
\end{Verbatim}
  The \texttt{-r} option means ``reboot.''

\end{enumerate}

When you are done with the steps above you can shut down DevBox and HackBox using a command such
as:
\begin{Verbatim}
$ sudo shutdown -h now
\end{Verbatim}
Here the \texttt{-h} option means ``halt.'' After the system shuts down you can restore to the
snapshot to undo any changes made during this session and put the system back into its initial
state. You may find it useful to do this after each experiment.\footnote{Avoid storing anything
of importance on HackBox. Use DevBox for all files you wish to keep, and even then back them up
regularly by transferring important files elsewhere.}

You should also restore the file \filename{version.c} in the source code to its original form.
This step isn't strictly necessary in this case since this is a minor, inconsequential change.
However, in the future you may be making non-trivial changes to the kernel source, and you'll
want an easy way to back them out when you are done (or if they cause extreme difficulties).

Although you will be recompiling the kernel in some labs, most of the programming you'll be
doing will actually be in the form of external kernel modules. These are modules that are not
part of the normal distribution and that are always loaded dynamically. Details about how to do
this will be provided in the appropriate lab.

\section{Phoenix}

\textit{This section needs to be reviewed and updated!}

In addition to Linux, we may also experiment with an operating system written by a team of
Vermont Technical College students for their 2008/2009 senior project. That system is called
Phoenix. Using Phoenix does not require HackBox at all. Instead, it runs inside a virtualization
product called Bochs running inside the DevBox VM. Yes, this amounts to a nested virtual
machine, but unlike the situation with Hyper-V described earlier Bochs is an ordinary user
process and doesn't introduce the same complications of trying to use a type 1 and type 2
hypervisor at the same time.

If you wish to try Phoenix to see if it works for you, follow the steps below.

\begin{enumerate}

\item On DevBox change to the \texttt{Projects/Phoenix} directory and do:
\begin{Verbatim}
$ git pull
\end{Verbatim}
  to get the most recent version of Phoenix from GitHub.

\item Type:
\begin{Verbatim}
$ source useOW.sh
\end{Verbatim}
  to configure the environment of your terminal to make the Open Watcom C/C++ compilers
  available. Phoenix is built with Open Watcom.

\item Use the \filename{makeandrun.sh} script in the top level Phoenix folder. That script
  creates a Phoenix boot disk image containing the system and several sample programs, and then
  launches Phoenix using the Bochs simulator. Bochs has been configured with debugging features
  turned on so when it starts it will produce a command prompt. Type ``continue'' to begin the
  simulation without interruption.

  When Phoenix boots you will be prompted with a menu of programs on the boot disk. Select one
  of them to see Phoenix in action. You can power off the Bochs simulation (upper right corner
  of the Bochs window) when you are done. There is no shutdown procedure for Phoenix.

\end{enumerate}

\section{Using KGDB and KDB}

\textit{This section needs to be reviewed and updated!}

KGDB is a kernel debugger that allows you to connect \command{gdb} running on DevBox to the
kernel running on HackBox. Using this tool is a bit intricate. Here I outline the basic
procedure. See other resources for additional information.

When you want to use the debugging tools you must log into HackBox by way of a serial terminal.
Do \emph{not} use SSH and do not use the console. While the HackBox kernel is being debugged the
network will not function normally. Furthermore, the debugger on DevBox has no way to transmit
data over the console; it needs to use an old style serial port.

\emph{I recommend taking a snapshot of HackBox before your debugging session in case you
  accidentally trash HackBox during the session!}

On DevBox use the command:
\begin{Verbatim}
$ minicom
\end{Verbatim}
This command starts the terminal program. It has been preconfigured to connect to
\filename{/dev/ttyS0} which is attached to HackBox's serial port of the same name by way of a
host-level named pipe. You can hit the Enter key to get a login prompt from HackBox. Log in as
the user student.

When you are ready to debug, you must first activate KGDB on HackBox. Use a command such as:
\begin{Verbatim}
# echo ttyS0 > /sys/module/kgdboc/parameters/kgdboc
\end{Verbatim}
This informs the KGDB ``over console'' driver that it should use ttyS0 for communication.
Without this step KGDB will effectively be turned off even if it has been compiled into the
kernel. Note that this command must be executed at a root shell (using \texttt{sudo} won't
work). You can, however, get a root shell by using the command \command{sudo bash}.

There are three ways to break into the kernel.
\begin{enumerate}
\item If the kernel takes an exception it will stop and wait for the debugger to attach.
\item You can issue a SysRq-g sequence in the terminal program.
\item You can echo the letter `g' into the file \texttt{/proc/\-sysrq-trigger}.
\end{enumerate}

The second option can be done from Minicom by typing Ctrl+Afg. Use the following command to
execute the third option:
\begin{Verbatim}
# echo g > /proc/sysrq-trigger
\end{Verbatim}

After breaking into the kernel you should see a KDB prompt on your terminal. At this point the
kernel on HackBox is frozen awaiting your debugging pleasure. You could enter KDB commands, but
it is generally more interesting to use \command{gdb}.

On DevBox in a separate command window, go into the top level of the Linux source tree and use
the command:
\begin{Verbatim}
$ gdb ./vmlinux
\end{Verbatim}
This runs the debugger against the previously compiled kernel image. The debugger will use the
image for symbolic information. Note that you must use the uncompressed kernel image and not
\texttt{vmlinuz}.

Connect the debugger to the ttyS0 serial port with the following commands:
\begin{Verbatim}
(gdb) set serial baud 115200
(gdb) target remote /dev/ttyS0
\end{Verbatim}

You can now use \texttt{gdb} debugging commands to debug the HackBox kernel. Note that you will
see some ``junk'' appearing on the previously opened terminal. They are remote debugging
protocol packets; you can ignore them.

To return to normal operation first exit \command{gdb} with the command:
\begin{Verbatim}
(gdb) quit
\end{Verbatim}
You will see a message about terminating the remote process. You can say `yes' here (it doesn't
actually terminate the HackBox kernel). Then, back in the terminal window type:
\begin{Verbatim}
$3#33
\end{Verbatim}
You won't be able to see the text as you type it. This sequence is a debugging protocol packet
that tells KDB to return to the prompt. You can then resume normal operation of HackBox by
issuing the \command{go} command at the KDB prompt.

After shutting down HackBox, consider restoring to the snapshot you made earlier to ensure the
system is in a consistent (non-destroyed) state.

\section{Making Backups}

Most recovery operations such as restoring to a snapshot or re-importing the original OVA file
will entail the loss of some or all of your work. Thus, I strongly recommend that you back up
your work often. This can be done by using a special backup script. To use it simply execute
\command{backup} at a command prompt.

It is extremely important to understand that The \command{backup} script only backs up the files
in the \filename{cis-4020} folder. This means the backup archives are relatively small, but it
also means that any files you put elsewhere on the system will not be backed up. \emph{I
therefor strongly recommend that you store all course materials in the cis-4020 folder.} Be
aware that any system configuration changes you make are not backed up.

Once the \command{backup} script has created the archive it will ask you if you want to transfer
that archive to Lemuria. Assuming you have an account on that system you can use Lemuria as a
repository of backups. If you do not transfer the backup file, the script will leave it in
student's home directory where you can transfer it some other way (for example, using WinSCP on
your host).

You should definitely transfer the backup archives off the virtual machine. The point of the
backups are to save your work in case the VM is destroyed. Keeping the backup archives on the VM
won't help you if you lose the state of the VM itself.

Figure~\ref{fig:sample-backup} shows a sample backup session. Text entered by the user is show
in an italic font.

\begin{figure*}[t]
\begin{Verbatim}[fontsize=\small, frame=single, commandchars=\\\{\}]
student@devbox:~$ \textit{backup}
1) None
2) cis-4020
=> \textit{2}
Creating backup file for cis-4020...
Done
Transfer to lemuria? [y/n] \textit{y}
Username: \textit{pchapin}
Password: 
backup-cis-4020-2024-06-19.tar.gz   100%   38KB  38.3KB/s   00:00    
Transfer successful. Removing ~/backup-cis-4020-2024-06-19.tar.gz
student@devbox:~$
\end{Verbatim}
\caption{Sample Backup Session}
\label{fig:sample-backup}
\end{figure*}

Another, more elegant way to transfer a course directory to another host is to use the
\command{rsync} program. This program only copies files that have changed and is thus often
faster than transferring an entire archive (even a compressed archive). The \command{rsync}
program will add and remove files and directories on the target as necessary and, in archive
mode, it even copies file permissions and date/time information. First move to student's home
directory. Then do
\begin{Verbatim}
$ rsync -vaz --delete \
    cis-4020 username@lemuria.cis.vermontstate.edu:
\end{Verbatim}

The \command{--delete} option tells \command{rsync} to remove files on the target that are not in
the source. The \command{-z} option specifies compressed mode; this is particularly useful if you
are on a low speed network connection since it will reduce the amount of network traffic
required. See the \command{rsync} manual page for more information.

Note that the \command{rsync} command above uses ssh as the underlying transport. Thus, you will
be prompted for your password, however your password will not appear on the network unencrypted.
Note also that the \command{rsync} command above will create (or update) a \filename{cis-4020}
directory beneath your home directory on the remote host.

\section{Shutting Down}

When you are finished using the virtual machines \emph{do not just close the VirtualBox window!}
Closing VirtualBox is equivalent to pulling the power plug on a real machine. Instead, you
should shut down the guest operating system properly. As the root user, issue the command:
\begin{Verbatim}
# shutdown -h now
\end{Verbatim}

Alternatively you can suspend the virtual machine. On the VirtualBox menu (not inside the guest)
do ``Machine $\rightarrow$ Close\ldots'' In the dialog box that appears select ``Save the
machine state.'' The next time you start the virtual machine it will resume from the saved
state. This is usually quicker than booting the system from scratch.

\section{Setup Notes}

This section contains some notes on how DevBox and HackBox were initially configured. They are
intended to be helpful to anyone who wants to build the systems from scratch rather than
downloading the preconfigured systems. If you are only interested in using DevBox and HackBox,
you can ignore this section.

Here is a list of packages that need to be installed in DevBox to build the kernel, assuming a
Ubuntu 24.04 system:
\begin{enumerate}
  \item \texttt{gcc}
  \item \texttt{make}
  \item \texttt{gdb}
  \item \texttt{flex}
  \item \texttt{bison}
  \item \texttt{libncurses-dev}
  \item \texttt{libssl-dev}
  \item \texttt{libelf-dev}
\end{enumerate}

These additional packages are needed for a more full-featured experience:
\begin{enumerate}
  \item \texttt{minicom}
  \item \texttt{cscope}
  \item \texttt{universal-ctags}
  \item \texttt{emacs-nox}
\end{enumerate}

Note that the \texttt{emacs-nox} package is the version of Emacs that runs in a terminal.
However, it will also install the postfix mail transport agent as a dependency. During the
installation of postfix, you will be asked what kind of mail configuration you want. Choose
``Local only'' to create a configuration that will only deliver mail to users on DevBox and not
attempt to use the network. This can be changed later, if desired.

To complete the installation of Emacs, it is recommended to crate a suitable \filename{.emacs}
file. It would also be reasonable to create a suitable \filename{.vimrc} file for those who
prefer to use Vim.

It is convenient to allow the root user to use SSH to log in and copy files. To do this it is
first necessary to set a password for root since Ubuntu, by default, has a locked root account
(it relies on the administrative user using \command{sudo} to raise privilege when necessary).
On both DevBox and HackBox, use the following command:
\begin{Verbatim}
  $ sudo passwd root
\end{Verbatim}

Then edit the file \filename{/etc/ssh/sshd\_config} and change the line:
\begin{Verbatim}
  PermitRootLogin prohibit-password
\end{Verbatim}
to
\begin{Verbatim}
  PermitRootLogin yes
\end{Verbatim}

Finally, restart the SSH server with the command:
\begin{Verbatim}
  # systemctl restart ssh
\end{Verbatim}

Normally root access via SSH is disallowed for security reasons. However, in a virtual machine
environment it is often convenient to have this capability, and the security concerns area
greatly reduced by the isolated nature of the environment.

\end{document}
