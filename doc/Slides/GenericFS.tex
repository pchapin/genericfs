%
% This is a presentation on the GenericFS project in general.
% Think of it as an overview.
%

\documentclass[landscape]{slides}
\usepackage{amsmath}
\usepackage{amssymb}
\usepackage{amstext}
\usepackage{color}
\usepackage{fancyvrb}
\usepackage[pdftex]{graphicx}
\usepackage{listings}
\usepackage{times}
\usepackage{url}

\input{slide-macros}

% Default settings for code listings.
\lstset{
  language=C,
  basicstyle=\ttfamily,
  stringstyle=\ttfamily,
  commentstyle=\ttfamily,
  xleftmargin=0.25in,
  showstringspaces=false}

\title{\color{titlecolor}GenericFS Overview}
\author{
  \begin{tabular}{c}
  \\[3mm]
  \Large{Peter Chapin} \\[2mm]
  \normalsize{Vermont State University}\\[5mm]
  \includegraphics[scale=0.80]{VTSU.png}\\[16mm]
  \end{tabular}
}
\date{October 22, 2024}

\begin{document}

\color{Black}
\pagecolor{Background}

\maketitle

%%%%%

\startslide{What is GenericFS?}
\begin{citemize}
  \item A simple file system driver for Linux.
    \begin{citemize}
    \item Developed for educational purposes.
    \item Documentation given high priority.
    \item Serves as a starting point for other file system projects.
    \end{citemize}
  \item Provides no special features.
    \begin{citemize}
    \item Just ``generic'' POSIX-style file system.
    \item No optimizations attempted.
    \end{citemize}
\end{citemize}
\stopslide

%%%%%

\startslide{Obtaining GenericFS}
Check out from the project's GitHub site

\vspace{5mm}
\centerline{\url{https://github.com/pchapin/genericfs.git}}

\begin{citemize}
\item Source code in C.
\item Documentation (mostly) in \LaTeX.
\item Code base currently assumes kernel version 6.9.3.
\end{citemize}
\stopslide

%%%%%

\startslide{Project Organization}
\begin{citemize}
\item Kernel Module
\begin{citemize}
\item Once loaded, kernel will understand GenericFS partitions.
\item Provides \texttt{/proc} interface to file system statistics.
\end{citemize}

\item Disk Tool
\begin{citemize}
\item Used to initialize (format) GenericFS partitions.
\item Used to ``manually'' create GenericFS data structures on disk.
\item Used to inspect GenericFS partitions.
\item Used to check and repair GenericFS partitions.
\end{citemize}
\end{citemize}
\stopslide

%%%%%

\startslide{Documentation}
GenericFS extensively documented, supporting its \cemph{educational mission}.

\begin{citemize}
\item General file system concepts.
\item GenericFS disk layout and design.
\item General kernel module development and testing methods.
\item Kernel file system internals.
\item Possible GenericFS extensions for student projects.
\end{citemize}
\stopslide

%%%%%

\startslide{Future Directions}
\begin{citemize}
\item Release to the public.
\begin{citemize}
\item Potentially useful at other schools.
\item Interesting to others writing file system modules.
\end{citemize}

\item Implement some extensions.
\begin{citemize}
\item Can code be modularized so that extensions can be mixed and matched?
\end{citemize}

\item Drivers for other systems. Windows? QNX?
  Phoenix\footnote{\url{https://github.com/pchapin/phoenix}}?
\item Instrument for file system debugging.
\item High integrity version with proved properties.
\end{citemize}
\stopslide

%%%%%

\startslide{Contact Information}
\centerline{Peter Chapin}
\makeatletter
\centerline{peter.chapin@vermontstate.edu}
\makeatother
\centerline{\url{https://github.com/pchapin/genericfs}}
\stopslide

\end{document}
