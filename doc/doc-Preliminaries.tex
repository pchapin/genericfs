
\section{Preliminaries}
\label{sec:preliminaries}

It is not my intention to discuss kernel programming in detail here. There are other sources of
information about that subject that you could consult, including the \textit{Linux Kernel
Development} guide that is included in the \GenericFS\ documentation set. However, I do want to
talk about the mechanics of compiling and testing \GenericFS. If you are only interested in
installing \GenericFS\ and do not intend to modify or enhance it, you should still read the
subsection below on compiling \GenericFS. Otherwise, you can skip the rest of this section.

\subsection{Compiling \GenericFS}
\label{sec:compiling}

\GenericFS\ comes in two parts: a kernel module that contains the file system driver and a
``disk tool'' that can be used to initialize (format), read, edit, and check \GenericFS\
partitions. These two components are compiled individually.

The module build process is highly integrated with the kernel build system. You need to use a
properly constructed Makefile if you are to build a module consistently. The Makefile provided
with the \GenericFS\ module assumes that you will be compiling the driver for use with the same
kernel that is executing at the time you do the compilation. If this is not the case you will
need to modify the Makefile to reflect your intended paths. Furthermore, you must first
configure the kernel sources against which you are building your module. More information about
configuring the kernel and compiling modules can be found in the \textit{Linux Kernel
Development} companion document.

To compile \GenericFS\ change into the \filename{driver} directory and issue the command
\command{make}. The driver will be left in the file \filename{genericfs.ko}.

To compile the disk tool change into the \filename{disktool} directory and issue the command
\command{make}. The disk tool will be left in the executable file \filename{disktool}.

\subsection{Setting Up a \GenericFS\ Partition}
\label{sec:partition}

To test \GenericFS\ you will need a free disk partition. The size is not critical, although it
would be nice if it was large enough to exercise the file system reasonably. You can put
\GenericFS\ on a physical disk partition, or you can prepare a file system image stored in an
ordinary file. The second method is more flexible and probably the best choice for
experimentation.

\subsubsection{Physical Partition}

Use the \command{fdisk} command to create a partition on a physical disk or on a virtual disk
created by virtualization software such as VirtualBox. You may have to reboot your machine after
creating the new partition, so the kernel will see the new partition table. Look for the device
files for the new disk and its partitions in the \filename{/dev} directory.

\todo{Describe the process of using \command{fdisk} in more detail.}

\subsubsection{File System Image}

If you do not have any unpartitioned disk space on your system, you can still create a partition
for \GenericFS\ by using the loop back driver. This allows you to treat an ordinary file like a
block device. Once configured you can make a \GenericFS\ file system inside this file just as
you would on any other partition. Furthermore, once you load the \GenericFS\ driver you'll be
able to mount the file system inside this file as well. Another advantage to this approach is
that you can copy the file containing the \GenericFS\ partition to another computer for
inspection or experimentation. This is easier than moving physical (or even virtual) disk drives
around.

To create an image file, first create a file of some suitable size using the \command{dd}
command. For example, to create a 192 MiB file use a command such as:
\begin{verbatim}
$ dd if=/dev/zero of=disk.img count=393216
\end{verbatim}

Here the input file is the special file \filename{/dev/zero} which returns an endless stream of
zero bytes. The output file is \filename{disk.img} in this example, but it could have any name.
The count is given in units of 512 bytes, so a count of 393216 corresponds to an output file of
exactly 192 MiB.

The significance of this size is that it is large enough to cause the free maps to span multiple
blocks (specifically, 1.5 blocks each) without creating a disk image that is unduly large. This
is important for testing purposes. The details of \GenericFS's on-disk layout are discussed in
Section~\ref{sec:structure}.

\subsubsection{Formatting}

To create a \GenericFS\ file system on your new partition or image file you should use an
appropriate user mode tool. An interactive tool of this sort, called \command{disktool}, is
included in the \GenericFS\ package and is described in
Section~\ref{sec:implementation-disktool}. It plays the role of both \command{mkfs} and
\command{fsck}. It also gives you a way of creating and viewing \GenericFS\ data structures on
your partition and thus can be helpful for testing and debugging the \GenericFS\ driver.

Start \command{disktool} providing the name of the partition on the command line. This name is
either the device file representing the physical partition or the name of a previously created
image file, such as \filename{disk.img} as described above. The disk tool treats both kinds of
``partitions'' the same way. Use the Initialize menu option to format the partition.

The disk tool also sets every byte on the partition to 0x55. This takes time and is unnecessary
in general. However, it makes debugging file system data structures a little easier by erasing
any previous and potentially confusing information. Note that 0x55 is used instead of 0x00 so
the ``unused'' bytes stand out more in the hex dumps provided by \command{disktool}.

\subsubsection{Mounting}

Before \GenericFS\ can be used it is necessary to load the \GenericFS\ driver. In the
\filename{driver} directory, as root, use the command:
\begin{verbatim}
$ sudo insmod genericfs.ko
\end{verbatim}

This command inserts the \GenericFS\ module. You can check to see if the driver loaded correctly
by inspecting the contents of \filename{/proc/filesystems}. The ``genericfs'' file system should
be included in the list (probably at the bottom).

Once the image file is formatted and once the \GenericFS\ driver has been loaded into the
kernel, you can mount the image file, as root, using \emph{one} of the commands below
\begin{verbatim}
$ sudo mount /dev/partition gfs
$ sudo mount -o loop disk.img gfs
\end{verbatim}

The first command is appropriate if you created a physical partition for \GenericFS. The second
command is appropriate if you created an image file. It uses the loop back driver to mount the
file system inside the image file. Use only one of the two commands shown, depending on your
situation.

The name \filename{gfs} is the mount point. It should be a pre-existing empty directory. After
executing the mount command, you should be able to change into the \filename{gfs} directory and
use the files there normally.

\subsubsection{Unmounting}

Before you can manipulate a \GenericFS\ partition directly with the disk tool you must first
unmount it. This can be done, as root, with the command:
\begin{verbatim}
$ sudo umount gfs
\end{verbatim}

Here \filename{gfs} is the name of the mount point.

It is not necessary to unload the \GenericFS\ driver when you unmount a partition. You should be
able to unmount and then re-mount a partition multiple times without touching the driver.
However, if you make a change to the driver (and recompile it) you will need to unload and then
reload the driver before your change will be visible to your system. Be aware that all
\GenericFS\ partitions need to be unmounted before you can unload the driver. Unload the driver
with the command:
\begin{verbatim}
$ sudo rmmod genericfs
\end{verbatim}

Notice that you don't include the \filename{.ko} extension on the name when you unload the
driver. Technically you \command{insmod} a file but \command{rmmod} a driver. Thus,
\command{insmod} requires the name of a file, but \command{rmmod} only expects the driver's
name.

\subsubsection*{Exercises}

\begin{enumerate}

\item The \command{fdisk} program will assign partition type 83 (``Linux'') to any new partition
  by default. Is this really an appropriate type for an experimental file system?

\item What would be a reasonable size for an experimental \GenericFS\ partition? Justify your
  answer. You may need to consult information about the \GenericFS\ on-disk layout in
  Section~\ref{sec:structure}.

\item Set up a file system image file and try formatting the image file using a standard file
  system such as ext4. Try mounting it and copying some files to it.

\item Would there be any complications in putting a file for the loop back driver on a
  \GenericFS\ partition and then building a \GenericFS\ file system inside that file? Think
  about what happens inside the \GenericFS\ driver. Try it.

\end{enumerate}

\subsection{Debugging \GenericFS}
\label{sec:debugging}

Debugging kernel modules is difficult because they are not normal processes and thus can't be
controlled by a debugger. There are a number of techniques you can use, however.

\subsubsection{Basic Debugging Techniques}

The first technique is to simply include extra \code{printk} calls in your module. You can then
observe which of these calls are triggered by looking for the output in the system log file (or
in other places depending on the configuration of \command{syslog} and the priority level you
give the \code{printk}). This technique can be effective, but it won't help you if the functions
in your module are not getting called. Also, you should be sure to enclose your \code{printk}
statements with conditional compilation directives so that they can be turned on and off by
simply recompiling your module with appropriate options.

In fact, there is currently a facility in the \GenericFS\ source for producing and managing
debugging messages this way. The details are in \filename{global.h}, but the specific macros of
interest are shown in Figure~\ref{fig:debugging-macros}.

\begin{figure*}[htbp]
  \centering
  \begin{wbigbox}
\begin{lstlisting}{}
#define DEBUG_HEADER KERN_INFO "GenericFS DEBUG: %s: "

#if DEBUG_LEVEL == 0
  #define GENERIC_DEBUG(level, statement)
#else
  #define GENERIC_DEBUG(level, statement) \
    if (DEBUG_LEVEL >= level) { statement; }
#endif

#define ENTERED \
  GENERIC_DEBUG(1, printk(DEBUG_HEADER "Entered\n", __FUNCTION__))
\end{lstlisting}
  \end{wbigbox}
  \caption{\GenericFS\ Debugging Macros (in \filename{global.h})}
  \label{fig:debugging-macros}
\end{figure*}

Several ``debugging levels'' can be used with these macros. Level one is the default and is
intended to output messages when each significant \GenericFS\ function is entered. This is
easily accomplished by using the \texttt{ENTERED} macro at the top of each such function. A
debug level of zero removes the debugging messages entirely. Higher debug levels are intended to
trace progressively finer details in the execution of the module. This is implemented by using
the \texttt{GENERIC\-\_DEBUG} macro where appropriate. To change the debug level one must modify
its definition in \filename{global.h} and recompile the module. In a future version of the
\GenericFS\ driver, the debug level might be dynamically configurable.

With the debugging messages enabled you can gain insight on the operation of the \GenericFS\
driver by observing what happens when a particular user mode program executes. For example, to
help understand how \GenericFS\ services a particular system call, write a program that invokes
that system call and then see which functions (if any) in the \GenericFS\ driver get invoked.
The program \command{strace} can help you find precisely which system calls your test program
uses. You should use a tool like \command{strace} instead of just guessing what system calls are
invoked because many apparent ``system'' calls are actually library functions built on top of
the true system calls\footnote{POSIX does not specify which functions are system calls and which
  are library functions. POSIX only requires that all functions it specifies be supported.}.

\subsubsection{Kernel Debugger}

\todo{Talk about using the kernel debugger.}
