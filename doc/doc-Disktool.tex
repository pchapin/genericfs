
\section{Disktool}
\label{sec:disktool}

To facilitate testing, the \GenericFS\ distribution includes a user mode program that can modify
a disk partition directly to build various file system data structures. This tool allows you to
``manually'' construct a \GenericFS\ file system for use by the kernel module even if the kernel
module is not in a state where it can write \GenericFS\ structures reliably. The tool also helps
you debug the kernel module. After making a change to the module and then running a test case,
you can unmount the \GenericFS\ partition and use the disk tool to inspect the \GenericFS\
structures that the module created or modified.

The disk tool takes the name of a disk partition on the command line and presents a menu of
options using a curses-based interface. The options are mostly self-explanatory. Note that the
disk tool combines the functionality of a file system creation tool, a file system checking
tool, and a file system dumping tool, all in an interactive package. While this is nice for
experimenting with \GenericFS, stand-alone versions of these tools should also be written at
for better scriptability.

Note that the disk tool accesses the partition directly by opening the block device file.
\emph{Never run the disk tool on a mounted partition!} This approach bypasses the usual file
system handling code and lets the disk tool communicate directly with the disk cache. In this
way the tool can inspect and modify file system data structures in arbitrary ways.

\subsection{Measuring Partition Size}
\label{sec:disktool-size}

The disk tool uses an \code{ioctl} call on the open block device in order to find out the size
of the partition. The code in Figure \ref{fig:get-size} illustrates this.

\begin{figure*}[tp]
  \centering
  \begin{wbigbox}
\begin{lstlisting}{}
/* Determine the partition size in blocks. */
if (ioctl(fd, BLKGETSIZE, &size) >= 0) {
  block_count = size / (BLOCKSIZE / 512);
  printf("%ld blocks; %ld bytes (4K blocks)\n",
    block_count, block_count * BLOCKSIZE);
}
else {
  printf("Can't figure out partition size!\n");
  return 1;
}
\end{lstlisting}
  \end{wbigbox}
  \caption{Getting Partition Size}
  \label{fig:get-size}
\end{figure*}

Here \code{block_count} is a long integer representing the number of blocks on the partition.
\code{BLKGETSIZE} is a special number defined in \code{<sys/ioctl.h>}. The \code{ioctl} call
returns the size of the partition in 512 byte sectors. \code{BLOCKSIZE} is 4 KiB in our case.

Note that if the partition is not a multiple of 4 KiB the last few bytes of the partition will
be unused. The partition size is always a multiple of 512 bytes since it contains an integer
number of sectors.

You can also use a program such as \filename{od} directly on the partition's device file to view
the data in the partition. However, \filename{od} does not understand \GenericFS, of course, and
thus won't display the data in an easily understood manner. However, it can be useful for
debugging the disk tool itself. See the \filename{od} manual page for more information.

% TODO: These exercises don't entirely make sense at this location in the document.
\subsubsection*{Exercises}

\begin{enumerate}

\item The disk tool should never be used on a partition that has been mounted. Why not?

\item Enhance the current disk tool. For example

  \begin{enumerate}
    \item Enhance or extend one of the existing options.
    \item Implement one of the currently unimplemented options.
    \item Add a new option.
    \item Add the ability to edit \GenericFS\ data structures as well as
      display them.
  \end{enumerate}
  
\item Implement a disk tool program for some other file system such as ext2.

\item Implement stand-alone \GenericFS\ tools for making, checking, and dumping a \GenericFS\
  file system. Review the ext2 tools for doing this for ideas on what they might do and how they
  might work.

\end{enumerate}

\subsection{Checking the File System}
\label{sec:disktool-check}

The disk tool can check the file system for consistency. This is done in the
\code{verify_file_system} function. There are two layers of checks done. The lowest layer checks
consistency of the blocks. The higher layer checks the consistency of the inodes.

Roughly, the block checking proceeds as follows:

\begin{enumerate}
  \item \textit{Allocate}. Create an array of counters with one entry for each block on the
  disk.

  \item \textit{Initialize}. Initialize all counters to zero, and then increment the counters
  for the pre-allocated blocks (super block, inode free map, block free map, inode table).

  \item \textit{Scan}. For each inode in use (as indicated in the inode free map), increment the
  counters for all the blocks reachable by the inode, including any indirect blocks.

  \item \textit{Check}. Scan the block free map and verify that for each block that is in use,
  its counter is one (i.e., there are no allocated blocks that aren't part of any file or that
  are part of more than one file), and for each block that is not in use, its counter is zero
  (i.e., there are no ``free'' blocks that \emph{are} part of a file).
\end{enumerate}

Roughly, the inode checking proceed as follows:

\begin{enumerate}
  \item \textit{Allocate}. Create an array of counters with one entry for each inode on the
  disk.

  \item \textit{Initialize}. Initialize all counters to zero.
  
  \item \textit{Scan Directory}. Increment the counter for the inode of the root directory. Then
  scan the contents of the root directory. For every directory entry corresponding to a file,
  increment the counter for the inode associated with that file. For every directory entry
  corresponding to a directory, recursively scan that directory.

  \item \textit{Check}. Scan the inode free map and verify that for each inode that is in use,
  its counter is the same as the \code{nlinks} field in the inode (i.e., the inode agrees with
  the number of directory entries that actually reference it), and for each inode that is not in
  use, its counter is zero (i.e., no ``free'' inode is mentioned in any directory).
\end{enumerate}

The directory scanning is implemented with a function that takes the handle to the open
partition and the inode number of the directory to scan. It reads the directory's data blocks
and walks the linked list of directory entries. For each entry it takes the inode number from
that entry and fetches the inode itself to see if the entry refers to a file or a subdirectory.
For subdirectories, the function recursively calls itself.
